\documentclass[10pt]{amsart}
\usepackage[margin=0.7in]{geometry}
\usepackage{graphicx}
\usepackage{amsmath, amssymb, amsthm} 
\usepackage{lmodern}
\usepackage[T1]{fontenc}
\usepackage{fancyhdr}
\usepackage[usenames,dvipsnames]{xcolor} % If using tikz, put \usepackage{tikz} after this
\usepackage{pdfcolmk}
\usepackage{tikz-cd}
\usepackage{enumitem}

% Our commands + environments are in style.cls. 
% Using the line below, they can be imported into any .tex file.
\usepackage{../../style}

% Header 
\newcommand{\header}[2]{
    {\noindent
    {\Large \bf Hartshorne #1 Exercises: #2}
    \hfill 
    {\large Feiyang Lin and Luke Trujillo}
    \vspace{0.5cm}}
}

\begin{document}

\header{1.3}{Morphisms}

%% 
\begin{exercise}[3.1]
    \begin{enumerate}[itemsep=1pt]
    \item Show that any conic in $\aa^2$ is isomorphic either to $\aa^1$ or
        $\aa^1 - \{0\}$ \emph{(cf.\ Ex.~$1.1$)}.
    \item Show that $\aa^1$ is \emph{not} isomorphic to any proper open subset
        of itself. (This result is generalized by \emph{(Ex.~$6.7$)} below.)
    \item Any conic in $\pp^2$ is isomorphic to $\pp^1$.
    \item We will see later \emph{(Ex.\ $4.8$)} that any two curves are
        homeomorphic. But show now that $\aa^2$ is not even homeomorphic to
        $\pp^2$.
    \item If an affine variety is isomorphic to a projective variety, then it
        consists of only one point.
    \end{enumerate}
\end{exercise}


%% 
\begin{exercise}[3.2]
    A morphism whose underlying map on the topological spaces 
    is a homeomorphism need not be an isomorphism.
    \begin{enumerate}[itemsep=1pt]
        \item For example, let $\varphi : \aa^1  \to \aa^2$ be defined by 
        $t \mapsto (t^2,t^3)$. Show that $\varphi$ defines a
        bijective bicontinuous morphism of $\aa^1$ onto the curve $y^2 = x^3$, 
        but that $\varphi$ is not an isomorphism.

        \item For another example, let the characteristic of the base field $k$ be $p > 0$, and
        define a map $\varphi : \aa^1 \to \aa^2$ by $t \to t^p$. Show that $\varphi$ 
        is bijective and bicontinuous but not an isomorphism. 
        This is called the \emph{Frobenius morphism}.
    \end{enumerate}
\end{exercise}

\begin{solution}
    \begin{luke}
        \begin{enumerate}
            \item 
            \begin{description}[itemsep = 2pt]
                \item[Bijection]
                Since $k$ is an integral domain, this map is injective. In addition, the map 
                is surjective: if $(x_0, y_0) \in Z(y^2 - x^3)$ are nonzero, we set $t_0 = \frac{y_0}{x_0}$. 
                We then have that $\varphi(t_0) = (t_0^2, t_0^3) = (x_0, y_0)$.  
                \item[Closed]
                Let $Y = Z(T) \in \aa^1$ be a closed set, $T$ some family 
                of polynomials in $k[x]$. Establishing notation, write 
                \[
                    \psi: k[x, y] \to k[t] \qquad x \mapsto t^2, y \mapsto t^3
                \]
                Now observe that 
                \begin{align*}
                    \varphi(Y) 
                    = 
                    \bigSet{ (z_0^2, z_0^3)}{z_0 \in Y }
                    = 
                    \bigSet{ (z_0^2, z_0^3)}{f(z_0) = 0 \quad \forall f \in T }
                    = 
                    Z(y^2 - x^3, T')
                \end{align*}
                where $T' = \{ g \in k[x, y] \mid \psi(g) \in T \}$. Hence $\varphi$ is a closed. 
                \item[Continuous]
                Now we show this function is continuous. Let $W = Z(S) \cap Z(y^2 - x^3)$ where $S$ 
                is a family of polynomials in $k[x, y]$. Then 
                \[
                    \varphi^{-1}(W) = \bigSet{t_0 \in \aa^1}{ f(t_0^2, t_0^3) = 0 \quad \forall f \in T \cap (y^2 - x^3)}.
                \] 
                If $f(t_0^2, t_0^3) = 0$, then $\psi(f)(t_0) = 0$. 
                Hence $\varphi^{-1}(W) \subseteq Z(\psi(T))$. The alternative direction is immediate, and we have that 
                $\varphi^{-1}(W) = Z(\psi(T))$, so that $\varphi^{-1}(W)$ is closed. 
                \item[Morphism]    
                We show that this function is actually a morphism of varieties. 
                Consider a regular map $f: V \subset Z(y^2 - x^3) \to k$ on some open subset 
                $V \subset Z(y^2 - x^3)$. Let $P \in \varphi^{-1}(V)$. Since $f: V \to k$ is regular, 
                there exits an open set $U \subseteq V$ containing $\varphi(p)$ such that 
                \[
                    f(x, y) = \frac{g(x, y)}{h(x, y)} \qquad \forall (x, y) \in U.  
                \]
                Now observe that $\varphi^{-1}(U)$ is an open subset of $\varphi^{-1}(V)$ containing 
                $P$, and moreover that 
                \[
                    f \circ \varphi(t) = f(t^2, t^3) = \frac{g(t^2, t^3)}{h(t^2, t^3)} \qquad \forall t \in \varphi^{-1}(U).
                \]
                Hence, $f \circ \varphi$ is regular whenever $f$ is, making $\varphi$ a morphism of 
                varieties. 
            \end{description}
            
            We finally show that this is not an isomorphism. This is because 
            \[
                k[x, y]/(y^2 - x^3) \not\cong k[x] \implies A(Z(y^2- x^3)) \not\cong A(\aa^1). 
            \]
            and this is due to the fact that $\psi$ is not surjective (e.g., $x \not\in \im(\psi)$, although $\ker(\psi) = I(y^2 - x^3)$).
            Hence, these varieties are not isomorphic.
           

            

            

            
    
        \end{enumerate}    
    \end{luke}
\end{solution}

%% 
\begin{exercise}[3.3]
    \begin{enumerate}[itemsep=1pt]
        \item 
        Let $\varphi:X \to Y$ be a morphism. Then for each $P \in X$, $\varphi$ 
        induces a homomorphism of local rings 
        $\varphi_P^*: \cO_{\varphi(P), Y} \to \cO_{P, X}$.

        \item Show that a morphism $\varphi$ is an isomorphism if and only if $\varphi$ is a 
        homeomorphism, and the induced map $\varphi_P^*$ on local rings 
        is an isomorphism, for all $p \in X$. 

        \item Show that if $\varphi(X)$ is dense in $Y$, 
        then the map $\varphi_P^*$ is injective for all $p \in X$.
    \end{enumerate}
    \end{exercise}

\begin{solution}
    \begin{luke}
        \begin{enumerate}
            \item Given a morphism $\phi: X \to Y$ of varieties and a point $P \in X$, 
            we define
            \[
                \phi_P^* : \cO_{\phi(P), Y} \to \cO_{P, X} \qquad \big(U \subset Y, f: U \to k \big) \longmapsto 
                \big(\phi^{-1}(U), f \circ \phi: \phi^{-1}(U) \to k\big).
            \]
            \begin{description}[itemsep = 3pt]
                \item[Well-defined] Let $(U_1, f_1) \sim (U_2, f_2)$ be members of $\cO_{\phi(P), Y}$.
                Then $f_1 \cap f_2$ on $U_1 \cap U_2$, which implies that 
                $f_1 \circ \phi = f_2 \circ \phi$ on $\phi^{-1}(U_1 \cap U_2)$. 
                Hence $(\phi^{-1}(U_1), f_1 \circ \phi) \sim (\phi^{-1}(U_2), f_2 \circ \phi)$.

                \item[Local]
                Let $\idl{m}_{\phi(P)} = \{ (V, f) \in \cO_{\phi(P), Y} \mid f\circ \phi(p) = 0 \}$ be the 
                maximal ideal of $\cO_{\phi(P), Y}$. Then observe that 
                $\phi_P^{*}(\idl{m}_{\phi(P)}) = (\phi^{-1}(V), f \circ \phi) \in \idl{m}_P \subset \cO_{P, X}$.
                Hence, we see that $\phi_P^{*}(\idl{m}_{\phi(P)}) \subset \idl{m}_{P}$, so that 
                $\phi_P^*$ is a homomorphism of local rings. 
            \end{description}

            \item Let $\phi: X \to Y$ be a homeomorphism such that $\phi_P^{*}$ is an isomorphism 
            for each $P \in X$. Let
            $f: V \subseteq Y \to k$ be a regular function on $V$ containing $P$.
            Then $f$ corresponds via $\phi_P^*$ to the regular function 
            $f \circ \phi: \phi^{-1}(V) \to k$ on $\phi^{-1}(P)$. 
            Hence, $\phi$ is a morphism of varieties. By a similar argument, $\phi^{-1}$ is also a morphism of varieties. 
            As they are inverses of each other, we see that $\phi$ is an isomorphism of varieties. 
            The other direction is not difficult to check.

            \item 
            
        \end{enumerate}
    \end{luke}
\end{solution}


%% 
\begin{exercise}[3.4]
    Show that the $d$-uple embedding of $\pp^n$ (Ex. 2.12) is an isomorphism 
    onto its image.
\end{exercise}

\begin{solution}
    \begin{luke}
        We recall some notation. 
        \begin{itemize}[itemsep=2pt, label={\textbullet}]
            \item $M_i(x_0, \dots, x_n)$ is the $i$-th $d$-degree monomial (from $0$ to $N = {n + d \choose n} -1 $) 
            \item $\rho_d: \pp^n \to Z(\idl{a})\subset \pp^N$ sends $(a_0, \dots, a_n)$ to $(M_0(a_0, \dots, a_n), \dots, M_N(a_0, \dots, a_n))$ 
            \item $\theta: k[y_0, \dots, y_N] \to k[x_0, \dots, x_n]$ sends $y_i$ to $M_i(x_0, \dots, x_n)$.
        \end{itemize}
        From Exercise 2.12, we already know that $\rho_d: \pp^n \to Z(\idl{a}) \subset \pp^N$ 
        is a homeomorphism. We show that $\rho_d$ and $\rho_d^{-1}$ are morphisms of varieties 
        and are inverses of each other (as morphisms of varieties).

        \begin{itemize}[itemsep=2pt, label={\textbullet}]
            \item We show $\rho_d$ is a morphism of varieties. 
            Let $f: V \subseteq Z(\idl{a}) \to k$ be regular. Then for each $p \in V$, there exists 
            an open set $U \subseteq V$ containing $p$ such that $f(y_0, \dots, y_N) = g_1(y_0, \dots, y_N)/g_2(y_0, \dots, y_N)$
            on $U$, for some homogeneous polynomials $g_1, g_2$ of the same degree. 
            Consider the function $f \circ \rho_d: \rho_d^{-1}(V) \to k$. Observe that 
            $\rho_d^{-1}(U)$ is a subset of $\rho_d^{-1}(V)$ containing $\rho_d^{-1}(p)$, 
            and moreover that 
            \[
                f \circ \rho_d = \frac{\theta(g_1)}{\theta(g_2)} \text{ on all of } \rho_d^{-1}(U).
            \]
            Note that $\theta(g_1)$, $\theta(g_2)$ are homogeneous polynomials of the same degree, 
            and moreover that $\theta(g_2) \ne 0$ on any of $\rho_d^{-1}(U)$. 
            Hence, $f \circ \rho_d$ is regular, which implies that $\rho_d$ is a morphism of varieties. 
        
            \item We show $\rho_d^{-1}$ is a morphism of varieties. Let $m: V \subseteq \pp^n \to k$ be a
            regular function. Then for each $p \in P$, there exists some open set $U \subseteq V$ containing 
            $p$ and a pair of 
            homogeneous polynomials $h_1, h_2$ of the same degree such that 
            $m(x_0, \dots, x_n) = h_1(x_0, \dots, x_n)/h_2(x_0, \dots, x_n)$ on all of $U$. 
            
            Take $U_i = \pp^n - H_i$ such that $p \in U \cap U_i$. Let 
            $\alpha = |d - \deg(h_1)| = |d - \deg(h_2)|$.  
            Observe that 
            \[
                m = \frac{x_i^{\alpha}h_1(x_0, \dots, x_n)}{x_i^{\alpha}h_2(x_0, \dots, x_n)} = \frac{h_1'(x_0, \dots, x_n)}{h_2'(x_0, \dots, x_n)}.
            \]
            Both $h_1'$ and $h_2'$ are homogeneous polynomials 
            of degree $\alpha + \deg(h_1) = \alpha + \deg(h_2)$ which is divisible by $d$.
            Hence, we see that both of these polynomials are in the image of $\theta$. 
            Let $k_1, k_2 \in k[y_0, \dots, y_N]$ such that $\theta(k_1) = h_1'$, $\theta(k_2) = h_2'$.
            Then we see that 
            \[
                f \circ \rho_d^{-1} = \frac{k_1}{k_2} \text{ on all of } \rho_d(U \cap U_i).
            \]
            Thus we see that $f \circ \rho_d^{-1}: \rho_d(V) \subset Z(\idl{a}) \to k$ is a regular function, so that 
            $\rho_d^{-1}$ is a morphism of varieties. 
        \end{itemize}
        As both $\rho_d$ and $\rho_d^{-1}$ are morphisms 
        of varieties, and are homeomorphisms, we see that $\rho_d$ establishes the isomorphism 
        $\pp^n \cong Z(\idl{a})$, as desired. 
    \end{luke}
\end{solution}

%% 
\begin{exercise}[3.5]
    By abuse of language, we will say that a variety ``is affine'' if it is 
    isomorphic to an affine variety. If $H \subseteq \pp^n$ is any hypersurface, 
    show that $\pp^n - H$ is affine. 
    [\emph{Hint}: Let $H$ have degree $d$. Then consider the $d$-uple 
    embedding of $\pp^n$ in $\pp^n$ and use the fact that $\pp^n$ minus 
    a hyperplane is affine.]
\end{exercise}


%% 
\begin{exercise}[3.6]
    There are quasi-affine varieties which are not affine. For example, show that 
    $X = \aa^2 - \{(0,0)\}$ is not affine. [\emph{Hint}: Show 
    that $\cO \cong k[x,y]$
    and use (3.5). See (III, Ex. 4.3) for another proof.]
\end{exercise}


%% 
\begin{exercise}[3.7]  
    \begin{enumerate}[itemsep=1pt]
        \item Show that any two curves in $\pp^2$ have a nonempty intersection.
        \item More generally, show that if $Y \subseteq \pp^n$ is a projective variety of 
        dimension $\ge 1$, and if $H$ is a hypersurface, then $Y \cap H \ne \varnothing$ . 
        [\emph{Hint}: Use (Ex. 3.5) and (Ex. 3.1e). See (7.2) for a generalization.]
    \end{enumerate}
\end{exercise}


%% 
\begin{exercise}[3.8]
    Let $H_i$ and $H_j$ be the hyperplanes in $\pp^n$ defined by 
    $x_i = 0$ and $x_j = 0$, with $i \ne j$. Show that any regular function on 
    $\pp^n - (H_i \cap H_j)$ is constant. 
    (This gives an alternate proof of (3.4a) in the case $Y = \pp^n$.)
\end{exercise}


%% 
\begin{exercise}[3.9]
    The homogeneous coordinate ring of a projective variety is not invariant under 
    isomorphism. For example, let $X = \pp^1$, and let $Y$ be the 2-uple embedding of $\pp^1$ in $\pp^2$. 
    Then $X \cong Y$ (Ex. 3.4). But show that $S(X) \not\cong S(Y)$.
\end{exercise}


%% 
\begin{exercise}[3.10]
    \emph{Subvarieties}. A subset of a topological space is locally closed if it is an 
    open subset of its closure, or, equivalently, if it is the intersection of an open set with 
    a closed set.
    
    If $X$ is a quasi-affine or quasi-projective variety and $Y$ is an irreducible locally closed 
    subset, then $Y$ is also a quasi-affine (respectively, quasi-projective) variety, by virtue 
    of being a locally closed subset of the same affine or projective space. We call this the induced 
    structure on $Y$, and we call $Y$ a subvariety of $X$.
    
    Now let $\varphi:X \to Y$ be a morphism, let $X' \subseteq X$ and $Y' \subseteq Y$ be irreducible locally closed subsets 
    such that $\varphi(X') \subseteq Y'$. Show that $\varphi|_{X'} :X' \to Y'$ 
    is a morphism.
\end{exercise}


%% 
\begin{exercise}[3.11]
    Let $X$ be any variety and let $P \in X$. Show there is a 1-1 correspondence between
    the prime ideals of the local ring $\cO_P$ and the closed subvarieties of $X$ containing $P$.
\end{exercise}


%% 
\begin{exercise}[3.12]
    If $P$ is a point on a variety $X$, then $\dim \cO_P = \dim X$. [\emph{Hint}: Reduce to the
    affine case and use (3.2c).]
\end{exercise}


%% 
\begin{exercise}[3.13]
    \emph{The Local Ring of a Subvariety}. Let $Y \subseteq X$ be a subvariety. Let $\cO_{Y,X}$ be the set 
    of equivalence classes $\<U,f\>$ where $U \subseteq X$ is open, $U \cap Y \ne 0$, and $f$ is a regular 
    function on $U$. We say $\<U, f\>$ is equivalent to $\<V, g\>$ if $f = g$ on $U \cap V$. Show that 
    $\cO_{Y,X}$ is a local ring, with residue field $K(Y)$ and dimension $= \dim X - \dim Y$. It is the local ring 
    of $Y$ on $X$. Note if $Y = P$ is a point we get $\cO_{P}$ and if $Y = X$ we get $K(X)$. 
    Note also that if $Y$ is not a point, then $K(Y)$ is not algebraically closed, so in this way we 
    get local rings whose residue fields are not algebraically closed.
\end{exercise}


%% 
\begin{exercise}[3.14]
    \emph{Projection from a Point}. Let $\pp^n$ be a hyperplane in $\pp^{n+1}$ and let $P \in \pp^{n+1} - \pp^n$. 
    Define a mapping $\varphi : \pp^{n+ 1} - P \to \pp^n$ by $\varphi(Q) = $ the intersection of the unique line containing $P$ and $Q$ 
    with $\pp^n$.
    \begin{enumerate}[itemsep=1pt]
        \item Show that $\varphi$ is a morphism.
        \item Let $Y \subseteq \pp^3$ be the twisted cubic curve which is the image of the 
        3-uple embedding of $\pp^1$ (Ex. 2.12). If $t$,$u$ are the homogeneous coordinates on $\pp^1$, 
        we say that $Y$ is the curve given parametrically by $(x,y,z,w) = (t^3,t^2u,tu^2,u^3)$. 
        Let $P = (0,0,1,0)$, and let $\pp^2$ be the hyperplane $z = 0$. Show that the projection of 
        $Y$ from $P$ is a cuspidal cubic curve in the plane, and find its equation.
    \end{enumerate}
\end{exercise}


%% 
\begin{exercise}[3.15]
    \emph{Products of Affine Varieties}. Let $X \subseteq \aa^n$ and $Y \subseteq \aa^m$ be affine varieties.
    \begin{enumerate}[itemsep=1pt]
        \item Show that $X \times Y \subseteq A^{n+m}$ with its induced topology is irreducible. 
        [\emph{Hint}: Suppose that $X \times Y$ is a union of two closed subsets $Z_1 \cup Z_2$. Let 
        $X_i = \{x \in X \mid x \times Y \subseteq Z_i\}$, $i = 1,2$. 
        Show that $X = X_1 \cup X_2$ and $X_1$, $X_2$ are closed. Then $X = X_1$ or $X_2$ so $X \times Y = Z_1$ or $Z_2$.] 
        The affine variety $X \times Y$ is called the product of $X$ and $Y$. Note that its topology is in general
        not equal to the product topology (Ex. 1.4).
        \item Show that $A(X \times Y) \cong A(X) \otimes_k A(Y)$.
        \item Show that $X \times Y$ is a product in the category of varieties, i.e., show (i) the
        projections $X \times Y \to X$ and $X \times Y \to Y$ are morphisms, and (ii) given a variety $Z$, 
        and the morphisms $Z \to X$, $Z \to Y$, there is a unique morphism $Z \to X \times Y$ making a commutative 
        diagram 
        \begin{center}
            \begin{tikzcd}[row sep = 1.2cm]
                Z 
                \arrow[rr, dashed]
                \arrow[dr]
                \arrow[drrr, crossing over]
                &[-0.25cm]
                &
                X \times Y
                \arrow[dl, crossing over]
                \arrow[dr]
                &[-0.25cm]
                \\
                &
                X
                &
                &
                Y
            \end{tikzcd}
        \end{center}

        \item Show that $\dim X \times Y = \dim X + \dim Y$.
    \end{enumerate}
\end{exercise}


%% 
\begin{exercise}[3.16]
    \emph{Products of Quasi-Projective Varieties}. Use the Segre embedding (Ex. 2.14) to identify $\pp^n \times \pp^m$ 
    with its image and hence give it a structure of projective variety. Now for any two quasi-projective varieties 
    $X \subseteq \pp^n$ and $Y \subseteq \pp^m$, consider $X \times Y \subseteq \pp^n \times \pp^m$
    \begin{enumerate}[itemsep=1pt]
        \item Show that $X \times Y$ is a quasi-projective variety.
        \item If $X$, $Y$ are both projective, show that $X \times Y$ is projective.
        \item[{\bf *(c)}] Show that $X \times Y$ is a product in the category of varieties.
    \end{enumerate}
\end{exercise}

%% 
\begin{exercise}[3.17]
    \emph{Normal Varieties}. A variety $Y$ is normal at a point $P \in Y$ if $\cO_p$ is an integrally closed ring. 
    $Y$ is normal if it is normal at every point.
    \begin{enumerate}[itemsep=1pt]
        \item Show that every conic in $\pp^2$ is normal.
        \item Show that the quadric surfaces $Q_1$,$Q_2$ in $\pp^3$ given by equations $Q_1 :xy = zw$;
        $Q2 :xy = z^2$ are normal (cf. (II. Ex. 6.4) for the latter.)
        \item Show that the cuspidal cubic $y^2 = x^3$ in $\aa^2$ is not normal.
        \item If $Y$ is affine, then $Y$ is normal $\iff A(Y)$ is integrally closed.
        \item Let $Y$ be an affine variety. Show that there is a normal affine variety $Y$, and a
        morphism $\pi: \tilde{Y} \to Y$, with the property that whenever $Z$ is a normal variety, and 
        $\varphi : Z \to Y$ is a dominant morphism (i.e., $\varphi(Z)$ is dense in $Y$), then there is a unique 
        morphism $e:Z \to \tilde{Y}$ such that $\varphi = \pi \circ \theta$. $\tilde{Y}$ is called the 
        normalization of $Y$. You will need (3.9A) above.
    \end{enumerate}
\end{exercise}

%% 
\begin{exercise}[3.18]
    \emph{Projectively Normal Varieties}. A projective variety $Y \subseteq \pp^n$ is projectively normal 
    (with respect to the given embedding) if its homogeneous coordinate ring $S(Y)$ is integrally closed.
    \begin{enumerate}[itemsep=1pt]
        \item If $Y$ is projectively normal, then $Y$ is normal.
        \item There are normal varieties in projective space which are not projectively normal. For example, let $Y$ be the 
        twisted quartic curve in $\pp^3$ given parametrically by $(x,y,z,w) = (t^4 ,t^3u,tu^3,u^4)$. Then $Y$ is normal 
        but not projectively normal. See (III, Ex. 5.6) for more examples.
        \item Show that the twisted quartic curve $Y$ above is isomorphic to $\pp^1$, which is projectively normal. 
        Thus projective normality depends on the embedding.
    \end{enumerate}
\end{exercise}

%% 
\begin{exercise}[3.19]
    \emph{Automorphisms of $\aa^n$}. Let $\varphi: \aa^n \to \aa^n$ be a morphism of $\aa^n$ to $\aa^n$ given 
    by $n$ polynomials $f_1, ... ,f_n$ of $n$ variables $x_1, ... ,x_n$ Let $J = \det [\partial f_i/\partial x_j] $ 
    be the \emph{Jacobian} polynomial of $\varphi$.
    \begin{enumerate}[itemsep=1pt]
        \item If $\varphi$ is an isomorphism (in which case we call $\varphi$ an automorphism of $\aa^n$) show
        that $J$ is a nonzero constant polynomial.
        \item[{\bf**(b)}] The converse of {\bf (a)} is an unsolved problem, even for $n = 2$. See, for example.
        Vitushkin [1].
    \end{enumerate}
\end{exercise}

%% 
\begin{exercise}[3.20]
    Let $Y$ be a variety of dimension $\ge 2$, and let $P \in Y$ be a normal point. Let $f$ be a regular function 
    on $Y - P$.
    \begin{enumerate}[itemsep=1pt]
        \item Show that $f$ extends to a regular function on $Y$.
        \item Show this would be false for $\dim Y = 1$.
        See (III, Ex. 3.5) for generalization.
    \end{enumerate}
\end{exercise}

\begin{exercise}[3.21]
    \emph{Group Varieties}. A group variety consists of a variety $Y$ together with a morphism $\mu: Y \times Y \to Y$, 
    such that the set of points of $Y$ with the operation given by $\mu$ is a group, and such that the inverse 
    map $y \to y^{-1}$ is also a morphism of $Y \to Y$.
    \begin{enumerate}[itemsep=1pt]
        \item The \emph{additive group} $\mathbf{G_a}$ is given by the variety $\aa^1$ 
        and the morphism $\mu: \aa^2 \to \aa^1$.
        defined by $\mu(a,b) = a + b$. Show it is a group variety.
        \item The multiplicative qroup $\mathbf{G_m}$ is given by the variety $\aa^1 - \{(0)\}$ and the 
        morphism $\mu(a, b) = ab$. Show it is a group variety.
        \item If $G$ is a group variety, and $X$ is any variety, show that the set $\Hom(X, \mathbf{G_a})$ has 
        a natural group structure.
        \item For any variety $X$, show that $\Hom(X,\mathbf{G_a})$ is isomorphic to $\cO(X)$ as a group
        under addition.
        \item For any variety $X$, show that $\Hom(X,\mathbf{G_m})$ is isomorphic to the group of units
        in $\cO(X)$, under multiplication.
    \end{enumerate}
\end{exercise}



\end{document}