\documentclass[10pt]{amsart}
\usepackage[margin=0.7in]{geometry}
\usepackage{graphicx}
\usepackage{amsmath, amssymb, amsthm} 
\usepackage{lmodern}
\usepackage[T1]{fontenc}
\usepackage{fancyhdr}
\usepackage[usenames,dvipsnames]{xcolor} % If using tikz, put \usepackage{tikz} after this
\usepackage{pdfcolmk}
\usepackage{enumitem}
\setlist{noitemsep}

% Our commands + environments are in style.cls. 
% Using the line below, they can be imported into any .tex file.
\usepackage{../../style}

% Header 
\newcommand{\header}[2]{
    {\noindent
    {\Large \bf Hartshorne #1 Exercises: #2}
    \hfill 
    {\large Feiyang Lin and Luke Trujillo}
    \vspace{0.5cm}}
}


\begin{document}

\header{1.2}{Projective Varieties}

% --
\begin{exercise}[2.1]
    Prove the ``homogeneous Nullstellensatz,'' which says if $\fa
    \subseteq S$ is a homogeneous ideal, and if $f \in S$ is a homogeneous
    polynomial with $\deg f > 0$, such that $f(P) = 0$ for all $P \in
    Z(\fa)$ in $\bbP^n$, then $f^q \in \fa$ for some $q >
    0$.
\end{exercise}

\begin{solution}
    \begin{luke}
    Let $\idl{a}$ be a homogeneous ideal of $k[x_0, \dots, x_n]$. 
    Then in $\bbP^n$, we have that  
    \[
        Z_{\pp^n}(\idl{a}) = \bigg\{ P \in \bbP^n \;\bigg|\; \text{For all } f \in \idl{a}, f(P) = 0  \bigg\}
    \]
    while in $\aa^{n+1}$, we have that 
    \[
        Z_{\aa^{n+1}}(\idl{a}) = \bigg\{ Q \in \aa^{n+1} \;\bigg|\; \text{For all } f \in \idl{a}, f(Q) = 0  \bigg\}.
    \]
    We can write a surjective map $\pi: Z_{\aa^{n+1}}(\idl{a}) \to Z_{\pp^n}(\idl{a})$ sending an affine point to its projective 
    equivalence class.
    In addition, for a projective point $P \in Z_{\pp^n}(\idl{a})$, we can observe that 
    $\pi^{-1}(P) \subset Z_{\aa^{n+1}}(\idl{a})$ (the elements of $P$'s equivalence class). 
    
    Thus, if $f$ is homogeneous with a nonzero degree and $f(P) = 0$ for all $P \in Z_{\pp^n}(\idl{a})$, 
    then $f(Q) = 0$ for all $Q \in Z_{\aa^{n+1}}(\idl{a})$. By the usual Nullstellensatz, 
    this implies that $f^q \in \idl{a}$ for some $q > 0$, which proves the result. 
    \end{luke}
\end{solution}

% --
\begin{exercise}[2.2]
    For a homogeneous ideal $\fa \subseteq S$, show that the following
    conditions are equivalent:
    \begin{itemize}
        \item[(\emph{i}.)] $Z(\fa) = \emptyset$ (the empty set);
        \item[(\emph{ii}.)] $\sqrt{\fa} =$ either $S$ or the ideal $S_+ =
        \bigoplus_{d > 0}S_d$;
        \item[(\emph{iii}.)] $\fa \supseteq S_d$ for some $d > 0$. 
    \end{itemize}
\end{exercise}

\begin{solution}
    \begin{luke}
        \begin{description}
            \item[$(i) \implies (ii)$] 
            Suppose $\idl{a}$ is a homogeneous ideal and $Z(\idl{a}) = \varnothing$. By the 
            homogeneous Nullstellensatz, it is vacuously true that for every 
            $f \in \oplus_{d > 0}S_d$, $f(P) = 0$ for all $P \in Z(\idl{a})$, and so 
            for every such $f$ there exists a $q > 0$ such that $f^q \in \idl{a}$. 
            Therefore, $\oplus_{d > 0}S_d \subseteq \sqrt{a}$. If $\idl{a}$ contains 
            a unit, then $\sqrt{\idl{a}} = S$. If $\idl{a}$ does not contain a unit, then 
            $\sqrt{\idl{a}} \subseteq \oplus_{d > 0}S_d \implies \oplus_{d > 0}S_d = \sqrt{\idl{a}}$.
    
            \item[$(ii) \implies (iii)$] 
            Suppose $\sqrt{a} = S$. Then this implies that $1 \in \idl{a} \implies \idl{a} = S$. Hence $S_d \subset \idl{a}$.
    
            Alternatively, suppose $\sqrt{a} = \oplus_{d > 0}S_d$. Then for every $f \in \oplus_{d > 0}S_d$, 
            there exists a $q > 0$ such that $f^q \in \idl{a}$. In particular, there 
            exist $r_1, \dots, r_n$ such that 
            \[
                x_1^{r_1}, \dots, x_n^{r_n} \in \idl{a}.
            \]
            Take $d = r_1 + \cdots + r_n$. We claim that $S_d \subset \idl{a}$. 
            To see this, note that every degree-$d$ homogeneous polynomial is of the form 
            \[
                \sum_{k \ge 0}c_k x_1^{\alpha_1(k)} \cdots  x_n^{\alpha_n(k)}
            \]
            where only finitely many summands are nonzero and $\alpha_1(k) + \cdots + \alpha_n(k) = d$ is a sum 
            of nonnegative integers. Since 
            $d = r_1 + \cdots + r_n$, we know that for any summand $x_1^{\alpha_1(k)} \cdots  x_n^{\alpha_n(k)}$
            at least one $\alpha_i(k) \ge r_i$. Hence, the summand is in $\idl{a}$, and so the whole 
            sum is in $\idl{a}$. Therefore $S_d \subseteq \idl{a}$. 
    
            \item[$(iii) \implies (i)$] 
            Suppose $S_d \subseteq \idl{a}$ for $d > 0$. Then $Z(\idl{a})$ must 
            at least contain points $P$ such that $f(P) = 0$ for all $f \in S_d$.
            This includes the polynomials $x_1^d, \dots, x_n^d$. However, these $n$ polynomials 
            cannot all be simultaneously zero. Hence $Z(\idl{a}) = \varnothing$. 
        
        \end{description}
    \end{luke}

    \lfy{
    (i) $\Rightarrow$ (ii). Since $Z(\fa) = \emptyset$, the zero set of $\fa$ in affine space \begin{luke}Did you mean projective space?\end{luke} is either $\emptyset$ or $\{\mathbf{0}\}.$ In the first case, we certainly have $\sqrt{\fa} = S$. In the second case, we have $\sqrt{\fa} = \{p \in S: p(0) = 0\} = S_+$.

    (ii) $\Rightarrow$ (iii). It suffices to show that there exists $d$ such that all monomials of degree $d$ lies in $\fa$. Since $S_+ \subset \sqrt{\fa},$ for each $0 \leq i \leq n$, there exists $d_i$ such that $x_i^{d_i} \in \fa$. Let $d = n\sum_{i=0}^{n}d_i$. Then if a monomial has degree $d$, then there exists $i$ such that the exponent of $x_i$ in the monomial is at least $\sum_{i=0}^{n}d_i$, and therefore at least $d_i$, which implies that the monomial is in $\fa$.

    (iii) $\Rightarrow$ (i). If $S_d \subseteq \fa$ for some $d > 0$, then $x_i \in \sqrt{\fa}$ for every $0 \leq i \leq n$, which means that $Z(\fa) = Z(\sqrt{\fa}) = \emptyset$. 
    }
\end{solution}

% --
\begin{exercise}[2.3]
    \begin{enumerate}
        \item If $T_1 \subseteq T_2$ are subsets of $S^h$, then $Z(T_1) \supseteq
        Z(T_2)$.
      \item If $Y_1 \subseteq Y_2$ are subsets of $\bbP^n$, then $I(Y_1) \supseteq
        I(Y_2)$.
      \item For any two subsets $Y_1,Y_2$ of $\bbP^n$, $I(Y_1 \cup Y_2) = I(Y_1)
        \cap I(Y_2)$.
      \item If $\fa \subseteq S$ is a homogeneous ideal with
        $Z(\fa) \ne \emptyset$, then $I(Z(\fa)) =
        \sqrt{\fa}$.
      \item For any subset $Y \subseteq \bbP^n$, $Z(I(Y)) = \overline{Y}$.
    \end{enumerate}
\end{exercise}

\begin{solution}
    \begin{luke}
        I did {\bf (a) -- (c)}, but I don't feel like \TeX-ing them.
        \begin{enumerate}
            \setcounter{enumi}{3}
            \item Suppose $f$ is homogeneous and $f \in I(Z(\idl{a}))$. Then 
            $f(P) = 0$ for all $P \in Z(\idl{a})$. By the homogeneous Nullstellensatz, 
            we see that $f \in \sqrt{\idl{a}}$.
            
            Suppose $f \in \idl{a}$. Then $f^q \in \idl{a}$ for some $q > 0$. 
            As $f^q(P) = 0$ for all $P \in Z(\idl{a})$, $f(P) = 0$ for all 
            $P \in Z(\idl{a})$ (as $k$ is an integral domain) and so it 
            follows that $f \in I(Z(\idl{a}))$.
            Our total work then shows that $I(Z(\idl{a})) = \sqrt{\idl{a}}$, as desired.
    
            \item
            We first know that $Y \subset Z(I(Y))$; we show that it is the smallest closed 
            set containing $Y$. Thus
            let $Z(J)$ be a closed set where $Y \subset Z(J) \subset I(Z(Y))$. 
            Then $I(Z(J)) \subset I(Y)$. Since $I(Z(J)) = \sqrt{J}$, we see that 
            $J \subset I(Y)$. Therefore, 
            $Z(I(Y)) \subset Z(J)$. Hence, $Z(I(Y)) = Z(J)$, and so $Z(I(Y)) = \overline{Y}$.  
        \end{enumerate}
    \end{luke}
\end{solution}

% --
\begin{exercise}[2.4]
    \begin{enumerate}
        \item There is a one-to-one inclusion-reversing correspondence between
        algebraic sets in $\bbP^n$ and homogeneous radical ideals of $S$ not equal
        to $S_+$ given by $Y \mapsto I(Y)$ and $\fa \mapsto
        Z(\fa)$. \emph{Note:} Since $S_{+}$ does not occur in this correspondence, 
        it is sometimes called te \emph{irrelevant} maximal ideal of $S$. 
        
        \item An algebraic set $Y \subseteq \bbP^n$ is irreducible if and only if $I(Y)$ is a prime ideal. 
        \item Show that $\bbP^n$ itself is irreducible. 
    \end{enumerate}
\end{exercise}

\begin{solution}

\end{solution}

% --
\begin{exercise}[2.5]
    \begin{enumerate}
        \item $\bbP^n$ is a noetherian topological space. 
        \item Every algebraic set in $\bbP^n$ can be written uniquely as a finite
          union of irreducible algebraic sets, no one containing another. These are
          called its \emph{irreducible components}. 
    \end{enumerate}
\end{exercise}

\begin{solution}

\end{solution}

\begin{exercise}[2.6]
    If $Y$ is a projective variety with homogeneous coordinate ring $S(Y)$, show that $\Dim S(Y) = \Dim Y + 1$. 
    [\emph{Hint}: Let $\phi_i: U_i \to \aa^n$ be the homeomorphism of (2.2), 
    let $Y_i$ be the affine variety $\phi_i(Y \cap U_i)$ and let $A(Y_i)$ be its affine coordinate ring.
    Show that $A(Y_i)$ can be identified with the subring of elements of degree 0 of the 
    localized ring $S(Y)x_i$. Then show that $S(Y)_{x_i} \cong A(Y_i)[x_i, x_i^{-1}]$. 
    Now use (1.7), (1.8A), and (Ex 1.10), and look at the transcendence degrees. Conclude also that 
    $\Dim Y = \Dim Y_i$ whenever $Y_i$ is empty.]
\end{exercise}

\begin{solution}
    \begin{luke}
        We follow the hints. Note that 
        \[
            S(Y)_{x_i}
            =
            \bigg\{ \frac{f(x_0, \dots, x_n) + I_{\pp^n}(Y)}{x_i^j} \;\bigg|\; f + I_{\pp^n}(Y) \in S(Y) \bigg\}.
        \]
        In addition, an element of degree  0 will be of the form $(f + I(Y))/x_i^d$, where $\deg(f) = d$. 
        Now in this case, 
        \[
            \frac{f(x_0, \dots, x_n) + I_{\pp^n}(Y)}{x_i^d}
            =
            f\left( \frac{x_0}{x_i}, \dots, 1, \dots, \frac{x_n}{x_i} \right) + I_{\pp^n}(Y)
            =
            f(x_0, \dots, 1, \dots, x_n) + I_{\aa^n}(Y_i)
        \]
        where $1$ appears in the $i$-th coordinate. Therefore, the homogeneous 
        representative $f$ may be regarded as an ordinary polynomial which is 
        not in $I(Y_i)$ (i.e., it at least does not vanish on $Y$ if the $i$-th coordinate is 
        fixed to be nonzero). Hence, we may identify the 0-degree element 
        $(f + I(Y))/x_i^d$ as a member of $A(Y_i)$. 

        Alternatively, let $f + I_{\aa^n}(Y_i) \in A(Y_i)$. 
        Since $f$ does not vanish on $Y$ when $x_i$ is fixed to 1, we 
        

        We
        may homogenize the representative to 
        obtain the homogeneous polynomial $x_i^df\left( \frac{x_0}{x_i}, \dots, \frac{x_n}{x_i} \right)$, 
        where $d = \deg(f)$. Note that this can represent the
        degree $d$-element of $x_i^df\left( \frac{x_0}{x_i}, \dots, \frac{x_n}{x_i} \right) + I(Y)$ in 
        $S(Y)$. This in turn corresponds to the degree $0$-element of 
        $S(Y)_{x_i}$: 
        \[
            \frac{x_i^df\left( \frac{x_0}{x_i}, \dots, \frac{x_n}{x_i} \right) + I(Y)}{x_i^d}
            =
            f\left( \frac{x_0}{x_i}, \dots, \frac{x_n}{x_i} \right) + I(Y).
        \]
        In this way we identify $f + I(Y_i) \in A(Y_i)$ with a degree 0 element of $S(Y)_{x_i}$. 
        
        
        


        
    
    \end{luke}
\end{solution}

\begin{exercise}[2.7]
    \begin{enumerate}
        \item $\Dim \bbP^n = n$.
        \item If $Y \subseteq \bbP^n$ is a quasi-projective variety, then $\Dim Y =
          \Dim \overline{Y}$.
    \end{enumerate}
\end{exercise}

\begin{solution}
    
\end{solution}

\begin{exercise}[2.8]
    A projective variety $Y \subseteq \bbP^n$ has dimension $n-1$ if
    and only if it is the zero set of a single irreducible homogeneous polynomial
    $f$ of positive degree. $Y$ is called a \emph{hypersurface} in $\bbP^n$.   
\end{exercise}

\begin{solution}
    
\end{solution}

\begin{exercise}[2.9]
    If $Y \subseteq \aa^n$ is an affine variety, we identify $\aa^n$ with an open
    set $U_0 \subset \pp^n$ by the homeomorphism $\varphi_0$. Then we can speak of
    $\overline{Y}$, the closure of $Y$ in $\pp^n$, which is called the
    \emph{projective closure} of $Y$. 
   \begin{enumerate}
     \item Show that $I(\overline{Y})$ is the ideal generated by $\beta(I(Y))$,
       using the notation of the proof of $(2.2)$. 
     \item Let $Y \subset \aa^n$ be the twisted cubic of \emph{(Ex $1.2$)}. Its
       projective closure $\overline{Y} \subset \pp^n$ is called the \emph{twisted
       cubic curve} in $\pp^3$. Find generators for $I(Y)$ and $I(\overline{Y})$,
       and use this example to show that if $f_1, \ldots, f_r$ generate $I(Y)$,
       then $\beta(f_1), \ldots, \beta(f_r)$ do \emph{not} necessarily generate
       $I(\overline{Y})$. 
   \end{enumerate}
\end{exercise}

\begin{solution}
    \begin{enumerate}
        \item Since $\overline{Y} = Z(I(\phi_o Y))$, $I(\overline{Y}) = I(\phi_o Y)$. So it suffices to show that $I(\phi_o Y) = \beta(I(Y))$. We have $I(\phi_0 Y) \supseteq \beta(I(Y))$ because $\phi_o(Y) \subseteq Z(\beta(I(Y)))$. Conversely, we have $I(\phi_0 Y) \subseteq \beta(I(Y))$ because any point of $Y$ vanishes on $p(1,x_1,\dots, x_n)$ for $p \in I(\phi_o(Y))$. In other words, if $p \in I(\phi_o(Y))$, then $p(1,x_1,\dots, x_n) \in I(Y)$, which implies that $p \in \beta(I(Y))$.
        \item Recall that the twisted cubic is the variety $Y = \{(t,t^2,t^3): t \in k\}$. We claim that $I(Y) = J = (y-x^2,z-x^3).$ Since $Z(J) = Y$, it suffices to check that $J$ is radical. Notice that $k[x,y,z]/J \cong k[x]$ has no nilpotents, which shows that $J$ is radical. Let $f_1 = y-x^2$ and $f_2 = z-x^3$. We claim also that $\beta f_1 = wy-x^2$ and $\beta f_2 = w^2 z - x^3$ don't generate $I(\overline{Y}) = I(\phi_o(Y))$. Since $\phi_o(Y) = \{[1:t:t^2:t^3]: t\in k\},$ we have $wz-xy \in I(\overline{Y})$. But this cannot be generated by $\beta f_1$ and $\beta f_2$ since the only term involving $z$ in $\beta f_1$ and $\beta f_2$ is $w^2 z$.
    \end{enumerate}

       \begin{luke}
        \begin{enumerate}
            \item 
            Note that since $\overline{Y} =  Z(I_{\pp^n}(Y))$, we have that 
            $I_{\pp^n}(Y) = I(Z(I_{\pp^n}(Y))) = I_{\pp^n}(Y)$. Additionally, we 
            have that 
            \[
                \beta(I_{\aa^n}(Y)) \subset I_{\pp^n}(Y) = I_{\pp^n}(\overline{Y}).
            \]
            Hence it remains to show the other inclusion. To do so, note that 
            \begin{align*}
                I_{\pp^n}(Y) 
                &= 
                \{f \in S^h \mid f(1, a_1, \dots, a_n) \quad (a_1, \dots, a_n) \in Y\}\\
                &=
                \{f \in S^h \mid \alpha(f) \in I_{\aa^n}(Y)  \}.
            \end{align*}
            If $f \in I_{\pp^n}(Y)$, then $\alpha(f) \in I_{\aa^n}(Y)$, in which 
            case $\beta(\alpha(f)) \in \beta(I_{\aa^n}(Y))$. Since $\beta(\alpha(f)) = f$, we see 
            that $f \in \beta(I_{\aa^n}(Y))$. Therefore $\beta(I_{\aa^n}(Y)) = I_{\pp^n}(Y) = I_{\pp^n}(\overline{Y})$.  
            
            \item Since $Y = \{(t, t^2, t^3) \mid t \in k\}$, we know that in $\pp^n$
            \[
                Y = \{ (1, t, t^2, t^3) \mid t \in k \}.
            \]
            Recall from Ex 1.2 that $I_{\aa^n}(Y) = (x^2 - y, x^3 - z)$. 
            By \textbf{(a)}, we know that $I_{\pp^n}(\overline{Y}) = \beta(I_{\aa^n}(Y))$.
            Now observe that $xy - z \in I_{\aa^n}(Y)$, so that $\beta(xy-z) = xy - zx_0 \in I(\overline{Y})$. 
            However, this polynomial cannot be generated by the ideal 
            \[
                (\beta(x^2 - y), \beta(x^3 - z)) = (x^2 - yx_0, x^3 - zx_0^2).  
            \]
            Hence we see that $I(\overline{Y}) \ne (\beta(x^2 - y), \beta(x^3 - z))$. 

        \end{enumerate}
    \end{luke}
\end{solution}

\begin{exercise}[2.10]
    Let $Y \subset \bbP^n$ be a nonempty algebraic set, and let $\theta\colon
    \aa^{n+1} \setminus \{(0,\cdots,0\} \to \bbP^n$ be the map which sends the
    point with affine coordinates $(a_0, \cdots, a_n)$ to the point with
    homogeneous coordinates $[a_0 :\cdots: a_n]$. We define the \emph{affine cone}
    over $Y$ to be $$C(Y) = \theta^{-1}(Y) \cup \{(0,\cdots, 0)\}.$$ 
    \begin{enumerate}
        \item Show that $C(Y)$ is an algebraic set in $\aa^{n+1}$, whose ideal is equal to $I(Y)$, considered as an ordinary ideal in $k[x_0, \cdots, x_n]$. 
        \item $C(Y)$ is irreducible if and only if $Y$ is irreducible. 
        \item $\Dim C(Y) = \Dim Y +1$.
    \end{enumerate}
    Sometimes we consider the projective closure $\overline{C(Y)}$ of $C(Y)$ in
    $\bbP^{n+1}$. This is called the \emph{projective cone} over $Y$.
\end{exercise}

\begin{solution}
    \begin{luke}
        \begin{enumerate}
            \item 
            Let $J$ be a set of homogeneous polynomials. Note that 
            \[
                \theta^{-1}(Z_{\pp}(J)) = Z_{\aa^{n+1}}(J) \setminus\{(0,\dots,0)\}
            \]
            Thus we have that
            \[
                Z_{\aa^{n+1}}(J) = \theta^{-1}(Z_{\pp^n}(J)) \cup \{(0,\dots,0)\} = C(Z_{\pp^n}(J)).
            \]
            Hence if $Y$ is algebraic, so is $C(Y)$. In addition, if $J = I_{\pp^n}(Y)$, then 
            \[
                Z_{\aa^{n+1}}(I_{\pp^n}(Y)) = C(Z_{\pp^n}(I_{\pp^n}(Y))) = C(Y).
            \]
            Since $I_{\pp^n}(Y)$ is a radical ideal, this implies that $I(C(Y)) = I_{\pp^n}(Y)$, regarded 
            as an ideal of $k[x_0, \dots, x_n]$.

            \item Since $I_{\aa^{n+1}}(C(Y)) = I_{\pp^n}(Y)$, the result follows immediately.
            
            \item Consider a maximal chain of closed, irreducible subsets of $Y$ in $\pp^n$. 
            \[
                Y_0 \subset Y_1 \subset \cdots \subset Y_r = Y.
            \]
            By {\bf (a)} and {\bf (b)}, these are in bijection with closed, irreducible 
            subsets in $\aa^{n+1}$:
            \[
                C(Y_0) \subset C(Y_1) \subset \cdots \subset C(Y_r) = C(Y).
            \]
            We can extend the chain by appending the single point $\{(0,\dots, 0)\}$:
            \[
                \{(0, \dots, 0)\} \subset C(Y_0) \subset C(Y_1) \subset \cdots \subset C(Y_r) = C(Y).
            \]
            Hence, $\dim C(Y) = \dim(Y) + 1$. 

        \end{enumerate}
    \end{luke}
\end{solution}

\begin{exercise}[2.11]
    A hypersurface defined by a linear polynomial is called a \emph{hyperplane}. 
        \begin{enumerate}
            \item Show that the following two conditions are equivalent for a variety $Y \subset \bbP^n$: 
                \begin{itemize}
                \item[(\emph{i}.)] $I(Y)$ can be generated by linear polynomials
                \item[(\emph{ii}.)] $Y$ can be written as an intersection of hyperplanes. 
                \end{itemize}
                In this case we say that $Y$ is a \emph{linear variety} in $\bbP^n$.
            \item If $Y$ is a linear variety of dimension $r$ in $\bbP^n$, show that $I(Y)$ is minimally generated by $n-r$ linear polynomials 
            \item Let $Y,Z$ be linear varieties in $\bbP^n$, with $\Dim Y = r$, $\Dim Z = s$. If $r+s-n \geq 0$, then $Y \cap Z \neq \emptyset$. 
            Furthermore, if $Y \cap Z \neq \emptyset$, then $Y \cap Z$ is a linear variety 
            of dimension $\geq r+s-n$ (Think of $\aa^{n+1}$ as a vector 
            space over $k$, and work with its subspaces.)
        \end{enumerate}
\end{exercise}

\begin{solution}
    \begin{luke}
        \begin{enumerate}
            \item Let $H_i = Z(f_i)$ be hyperplanes, $f_i$ a linear homogeneous polynomial, for $i = 1, 2, \dots, r$. 
            Then 
            \begin{align*}
                Y 
                = H_1 \cap \cdots \cap H_r
                =
                Z(f_1) \cap \cdots \cap Z(f_r)
                \iff
                I(Y) 
                = I(Z(f_1) \cap \cdots Z(f_r))
                =
                I(Z(f_1, \dots, f_r)).
            \end{align*}
            However, since $f_1, \dots, f_n$ are homogeneous polynomials of degree 1, it is prime.
            Therefore, $I(Z(f_1, \dots, f_r)) = (f_1, \dots, f_r) \implies I(Y) = (f_1, \dots, f_r)$.
            Hence (\emph{i}.) and (\emph{ii}.) are equivalent.
            
            \item By Theorem 1.8A, we know that since $Y$ is a linear varitety, 
            \[
                \Ht(I(Y)) + \Dim( k[x_0, \dots, x_n]/I(r)) = n + 1
                \implies 
                \Ht(I(Y)) + r + 1 = n + 1
                \implies 
                \Ht(I(Y)) = n - r.
            \]
            By {\bf (a)}, we know that $I(Y)$ is generated by some linear homogeneous 
            polynomials $f_1, \dots, f_s$. This in turn creates a chain of prime ideals:
            \[
                (0) \subset (f_1) \subset (f_1, f_2) \subset \dots \subset (f_1, \dots, f_s) = I(Y).
            \]
            Since $\Ht(I(Y)) = n - r$, we see that some of the generators must be redundant, 
            so that $I(Y)$ is minimally generated by $n - r$ elements. 
            
            \item Since $r + s - n \ge 0$, we can conclude that $(n - r) + (n - s) \le n$. This implies 
            that $Y \cap Z$ is a system of equations with less equations than unknowns. From linear 
            algebra, we know that if such a system is homogeneous (i.e., set equal to zeroes), then 
            there are infinitely many solutions. Therefore, 
            \[
                Y \cap Z = Z(f_1, \dots, f_{n-s}) \cap Z(g_1, \dots, g_{n-r}) = Z(f_1, \dots, g_{n-r}) \ne \varnothing.                
            \] 
            In addition, we also see that this is a linear variety as claimed. Further, we know by {\bf (a)} 
            that $n - \dim(Y \cap Z) \le (n - r) + (n - s) \implies \dim(Y \cap Z) \ge r + s - n$, as desired.



        \end{enumerate}
    \end{luke}
\end{solution}

\begin{exercise}[2.12]
    For given $n, d>0$, let
    $M_0,\ldots, M_N$ be all the monomials of degree $d$ in the $n+1$ variables
    $x_0, \ldots x_n$, where $N = \binom{n+d}{n} -1.$ We define a mapping
    $\rho_d\colon \bbP^n \to \bbP^N$ by sending the point $P = (a_0, \ldots, a_n)$
    to the point $\rho_d(P) = (M_0(a), \ldots, M_N(a))$ obtained by substituting
    the $a_i$ in the monomials $M_j$. This is called the $d$-uple \emph{embedding}
    of $\bbP^n$ in $\bbP^N$. For example, if $n=1, d=2$, then $N= 2$, and the image
    $Y$ of the $2$-uple embedding of $\bbP^1$ in $\bbP^2$ is a conic. 
    \begin{enumerate}
      \item Let $\theta\colon k[y_0, \ldots, y_N] \to k[x_0, \ldots, x_n]$ be the
        homorphism defined by sending $y_i$ to $M_i$, and let $\mathfrak{a}$ be
        the kernel of $\theta$. Then $\mathfrak{a}$ is homogeneous prime ideal, and
        so $Z(\mathfrak{a})$ is a projective variety in $\bbP^N$. 
      \item Show that the image of $\rho_d$ is exactly $Z(\mathfrak{a})$.
      \item Now show that $\rho_d$ is a homeomorphism of $\bbP^n$ onto the projective
        variety $Z(\mathfrak{a})$. 
      \item Show that the twisted cubic curve in $\bbP^3$
        {\emph{(Ex.\ $2.9$)}} is equal to the $3$-uple
        embedding of $\bbP^1$ in $\bbP^3$, for suitable choice of coordinates. 
    \end{enumerate}
\end{exercise}

\begin{solution}
        
\begin{enumerate}
    \item 
    \item We prove the harder direction that $Z(\fa) \subseteq \im(\rho_d).$ We may index the $N+1$ coordinates of a point in $\bbP^n$ by tuples of the form $(a_0, a_1, \dots, a_n)$ where $a_i \in \Z_{+}$ and the sum of all $a_i$'s is $d$. Given $\y \in Z(\fa)$, I claim that there exists $0 \leq i \leq n$ such that $y_{d\bbe_i} \neq 0$. Indeed, suppose towards the contrary. Then since for any index $\bv = (a_0, a_1, \dots, a_n)$, $p(\y) = y_{\bv}^d - \prod_{i = 0}^{n}y_{d \bbe_i}^{a_i} \in \fa$, we have that $p(\y) = y_{\bv}^d = 0$, which implies that $y_{\bv} = 0$ for arbitrary $\bv$, which is absurd. 

    To give an example to make this proof clearer, consider the example when $n = 1$ and $d = 3$. Then if $\y = (y_{30}, y_{21}, y_{12}, y_{03}) \in Z(\fa)$, suppose WLOG that $y_{30} = 1$. Then $\y = \rho_d(1, y_{21})$. We check that since $y_{30}y_{12} - y_{21}^2 = 0$, indeed $y_{12} = y_{21}^2$. Similarly, since $y_{03}y_{30}^2 - y_{21}^3$, indeed $y_{03} = y_{21}^3$.

    \item The map $\rho_d$ is clearly a bijection between $\bbP^n$ and $\im \rho_d = Z(\fa)$. So it suffices to show that $\rho_d$ is bicontinuous, or equivalently, that it identifies the closed sets in $\bbP^n$ and $Z(\fa)$. 
        
    ($\rho_d$ continuous.) We claim that for any ideal $I \subset k[y_0, \dots, y_N]$, \[\rho_d^{-1}(Z(I)) = Z(\theta(I)).\] Notice that if $(x_0, \dots, x_n) \in \rho_d^{-1}(Z(I))$, then $p(M_0(\bx), \dots, M_N(\bx)) = 0$ for all $p(y_0, \dots, y_N) \in I$. If $(x_0, \dots, x_n) \in Z(\theta(I))$, then for all $p(y_0, \dots, y_N) \in I$, $\theta(p)(x_0, \dots, x_n) = 0$. But $\theta(p)(x_0, \dots, x_n) = p(M_0(\bx), \dots, M_N(\bx))$. So these two conditions are equivalent.

    ($\rho_d^{-1}$ continuous.) We claim that for any ideal $J \subset k[x_0, \dots, x_n]$, \[\rho_d(Z(J)) = Z(\theta^{-1}J) \cap Z(\fa).\] Indeed, if $\y = \rho_d(\x)$ where $\x \in Z(J)$, then for any $p \in \theta^{-1}J$, $p(\y) = p(\rho_d(\x)) = 0$ because $p \circ \rho_d = \theta(p) \in J$. Conversely, if $\y \in Z(\theta^{-1}J) \cap Z(\fa)$, then by part (ii), since $Z(\fa) = \im \rho_d$, there exists $\x$ such that $\y = \rho_d(\x)$. We check that $\x \in Z(J)$: for all $q$ such that $q \circ \rho_d \in J$, we have that $q(\y) = q(\rho_d(\x))= 0$. In other words, for all $p \in J$, $p(\x) = 0$.
\end{enumerate}
    

    \begin{luke}
        \begin{enumerate}
            \item First, it is prime since $\theta$ maps into an integral domain.
            Now recall that we may uniquely express
            any multivariate polynomial in $y_1, \dots, y_N$ into 
            homogeneous components. Since $y_i \not\in \ker(\theta)$, 
            no single monomial of any degree in $k[y_1, \dots, y_N]$ 
            maps to zero. Therefore, if
            for a multivariate polynomial $f$ we have that $\theta(f) = 0$, 
            then it must be that each 
            of the homogeneous components of $\theta(f_i)$ must map to zero, by 
            individually canceling each other out (in their own degree). 
            In other words, a polynomial is in $\idl{a}$
            if and only if its homogeneous components are in $\idl{a}$. 
            Thus, $\idl{a}$ is homogeneous. 

            \item ${\big[}\im(\rho_d) \subset Z(\idl{a}){\big]}$. 
            If $q \in \im(\rho_d)$, then $q = (M_0(P), \dots, M_N(P))$ for some $P \in \pp^n$, 
            and so for any $f \in \idl{a}$, we have that $f(q) = f(M_0(P), \dots, M_N(P)) = 0.$
            Therefore, $q \in Z(\idl{a})$.
            \\
            \\
            \noindent ${\big[}Z(\idl{a}) \subset \im(\rho_d){\big]}$. 
            Denote the coordinate $y_m$ to be 
            \[
                y_m =    M_m(x_0, \dots, x_n) = x_0^{\alpha_1^m} \cdots x_n^{\alpha_n^{m}}
            \]
            where $\alpha^m_i$ are nonnegative and sum to $d$. In particular, denote
            \[
                y_0 = x_0^d, \quad y_1 = x_1^d, \quad \dots, \quad y_n = x_n^{d}.
            \]
            We make some observations. 
            \begin{itemize}
                \item Observe that $y_m^d - y_0^{\alpha_0^m} \cdots y_n^{\alpha_n^{m}} \in \idl{a}.$
                The polynomial is nonzero when $m > n$. 
                Thus if $ Q = (b_0, \dots, b_n, b_{n+1}, \dots, b_{N}) \in Z(\idl{a})$, then for each $n < m \le N$, we see that 
                $b_m^d = b_0^{\alpha_0^m} \cdots b_n^{\alpha_n^{m}}$, so that at least one $b_0, \dots, b_n \ne 0$.


                \item Since one of $b_0, \dots, b_n \ne 0$, suppose that it is $b_0$. 
                Let $c \in \pp^N$ be such that $c_i = b_0^{d-1}b_i$. 
                Then observe that 
                \begin{align*}
                    M_m(c_0, \dots, c_n) 
                    &= (c_0)^{\alpha_1^{m}} \cdots (c_n)^{\alpha_n^{m}}\\
                    &= (b_0^{d-1} b_0)^{\alpha_1^{m}} \cdots (b_0^{d-1} b_n)^{\alpha_n^{m}}\\
                    &= (b_0^{d-1})^{d}(b_0)^{\alpha_1^m} \cdots(b_n)^{\alpha_n^m} \\
                    &= (b_0^{d-1})^d b_m.
                \end{align*}
                Hence,
                \begin{align*}
                    (b_0, \dots, b_n, \dots, b_N) 
                    &=
                    \big((b_0^{d-1})^{d}\cdot b_0, \dots, (b_0^{d-1})^{d} \cdot b_n, \dots, (b_0^{d-1})^{d} \cdot b_N\big)\\
                    &=
                    (M_0(c_0, \dots, c_n), \dots, M_n(c_0, \dots, c_n), \dots, M_N(c_0, \dots, c_n)).
                \end{align*}
                Therefore, $(b_0, \dots, b_N) \in \im(\rho_d)$. 
    
            \end{itemize}
            
        \item We show it is a homeomorphism. First, it is surjective onto $Z(\idl{a})$. 
        By our work from the last part, it is also injective, for we constructed the map 
        \[
            \rho_d^{-1}: Z(\idl{a}) \to \pp^n \qquad (b_0, \dots, b_n, b_{n+1}, \dots, b_N) \mapsto ((b_0^{d-1})^d b_0, \dots, (b_0^{d-1})^d b_n).
        \]
        We now show that $\rho_d$ is closed. Suppose $Y \subset \pp^n$ is closed, and that 
        $Y = Z(T)$ for some family of homogeneous polynomials in $(n+1)$-variables. 
        Then 
        \[
            \rho_d(Y) = \bigg\{ (M_0(P), \dots, M_N(P)) \;\bigg|\; P \in Y \bigg\}.
        \]
        Note that $\rho_d(Y) \subset Z(\idl{a})\cap Z(T)$. In addition, if $(b_0, \dots, b_N) \in Z(\idl{a}) \cap Z(T)$, 
        then $(b_1, \dots, b_n) \in Z(T) \subset \pp^n$. Now $b_1, \dots, b_n$ completely 
        determine the rest of the values $(b_{n+1}, \dots, b_N)$, so we have that 
        $Z(\idl{a}) \cap Z(T) \subset \rho_d(Z(T))$. Hence, we see that $\rho_d(Z(T)) = Z(\idl{a})\cap Z(T)$ is closed. 

        We now show $\rho_d$ is continuous. Let $Z(\idl{a}) \cap Z(T)$ be a closed set 
        with $T$ a family of homogeneous polynomials in $N+1$ variables. If $P = 
        (b_0, \dots b_n, \dots, b_N) \in Z(\idl{a}) \cap Z(T)$, then 
        \begin{align*}
            f(b_0, \dots, b_n, \dots, b_N) = 0
            &\implies 
            f(M_0(c_0, \dots, c_n), \dots M_N(c_0, \dots, c_n))\\
            &\implies 
            \theta(f)(c_0, \dots, c_n) = 0.
        \end{align*} 
        Hence we see that $Z(\theta(T)) \subset \rho^{-1}(Z(\idl{a}) \cap Z(T))$. 
        As the other direction is immediate, we see that $\rho^{-1}(Z(\idl{a}) \cap Z(T)) = Z(\theta(T))$ 
        and so $\rho_d$ is continous.

            
        \end{enumerate}
    \end{luke}
\end{solution}

\begin{exercise}[2.13]
    Let $Y$ be the image of the $2$-uple embedding of $\bbP^2$ in $\bbP^5$. This is the
    \emph{Veronese surface}. If $Z \subseteq Y$ is a closed curve (a \emph{curve} is 
    a variety of dimension $1$), show that there exists a hypersurface $V \subseteq
    \bbP^5$ such that $V \cap Y = Z$. 
\end{exercise}

\begin{solution}
    
\end{solution}

\begin{exercise}[2.14]
    Let $\psi\colon \bbP^r \times \bbP^s \to \bbP^N$ be the map defined by sending
    the ordered pair $(a_0, \ldots, a_r) \times (b_0, \ldots, b_s)$ to $(\ldots,
    a_ib_j, \ldots)$ in lexicographic order, where $N = rs + r +s$. Note that
    $\psi$ is well-defined and injective. It is called the \emph{Segre embedding}.
    Show that the image of $\psi$ is a subvariety of $\bbP^N$.
\end{exercise}

\begin{solution}
    
\end{solution}

\begin{exercise}[2.15]
    Consider the surface $Q$ (a \emph{surface} is variety of dimension $2$) in
    $\bbP^3$ defined by the equation $xy-zw = 0$.
    \begin{enumerate} 
      \item Show that $Q$ is equal to the Segre embedding of $\bbP^1 \times \bbP^1$ in
        $\bbP^3$, for suitable choice of coordinates.
      \item Show that $Q$ contains two families of lines (a \emph{line} is a
        linear variety of dimension $1$) $\{L_t\}, \{M_t\}$, each parametrized by
        $t \in \bbP^1$, with the properties that if $L_t \ne L_u$, then
        $L_t \cap L_u = \emptyset$; if $M_t \ne M_u$, $M_t \cap M_u = \emptyset$,
        and for all $t,u, L_t \cap M_u =$ one point. 
      \item Show that $Q$ contains other curves besides these lines, and deduce that
        the Zariski topology on $Q$ is not homeomorphic via $\psi$ to the product
        topology on $\bbP^1 \times \bbP^1$ (where each $\bbP^1$ has its Zariski
        topology). 
    \end{enumerate}
\end{exercise}

\begin{solution}
    
\end{solution}

\begin{exercise}[2.16]
    \begin{enumerate}
        \item The intersection of two varieties need not be a variety. For example,
        let $Q_1$ and $Q_2$ be the quadric surfaces in $\bbP^3$ given by the
        equations $x^2-yw = 0$ and $xy - zw =0$, respectively. Show that $Q_1 \cap
        Q_2$ is the union of a twisted cubic curve and a line.
        
        \item Even if the intersection of two varieties is a variety, the ideal of the
        intersection may not be the sum of the ideals. For example, let $C$ be the
        conic in $\bbP^2$ given by the equation $x^2-yz = 0$. Let $L$ be the line
        given by $y = 0$. Show that $C \cap L$ consists of one point $P$, but that
        $I(C) + I(L) \ne I(P)$. 
    \end{enumerate}
\end{exercise}

\begin{solution}
    
\end{solution}

\begin{exercise}[2.17]
    A variety $Y$ of dimension $r$ in $\bbP^n$ is a \emph{(strict) complete
    intersection} if $I(Y)$ can be generated by $n-r$ elements. $Y$ is a
    \emph{set-theoretic complete intersection} if $Y$ can be written as the
    intersection of $n-r$ hypersurfaces.
    \begin{enumerate}
    \item Let $Y$ be a variety in $\bbP^n$, let $Y = Z(\mathfrak{a})$; and
    suppose that $\mathfrak{a}$ can be generated by $q$ elements. Then show
    that $\Dim Y \ge n-q$. 
    \item Show that a strict complete intersection is a set-theoretic complete
    intersection.
    \item The converse of $(b)$ is false. For example let $Y$ be the twisted
    cubic curve in $\bbP^3$ {\emph{(Ex.\ 2.9)}}. Show that
    $I(Y)$ cannot be generated by two elements. On the other hand, find
    hypersurfaces $H_1,H_2$ of degrees $2,3$ respectively, such that
    $Y = H_1 \cap H_2$. 
    \item It is an unsolved problem whether every closed irreducible curve in
    $\bbP^3$ is a set-theoretic intersection of two surfaces.
    \end{enumerate}
\end{exercise}

\begin{solution}
\begin{enumerate}
    \item If $\fa = $
\end{enumerate}
\end{solution}

\begin{exercise}[3.15]
\end{exercise}

\begin{solution}
    \begin{enumerate}
        \item \lfy{As hinted, let $X_i = \{x \in X \mid x \times Y \subseteq Z_i\}$ for $i = 1,2$. We first prove that $X = X_1 \cup X_2$. If this doesn't happen, then there exists $y_1, y_2 \in Y$ such that $(x, y_1) \in Z_1 \setminus Z_2$, $(x, y_2) \in Z_2 \setminus Z_1$. However, this implies that $Z_1\cap x \times Y$ and $Z_2 \cap x \times Y$ are both (non-empty) proper subsets of $x \times Y$. But they are both closed, and $\{x\} \times Y$ is irreducible, which is a contradiction.}
    \end{enumerate}
\end{solution}

\end{document}