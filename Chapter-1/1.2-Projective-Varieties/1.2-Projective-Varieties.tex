% \documentclass{/Users/SHER/Documents/Hartshorne-Exercises/hw_pset} % hw_pset simply loads article.cls and creates the exercise/solution environments.
% \documentclass{hw_pset}
\documentclass[10pt]{amsart}
\usepackage[margin=0.7in]{geometry}
\usepackage{graphicx}
\usepackage{amsmath, amssymb} 
\usepackage{lmodern}
% \usepackage{tikz}
\usepackage[T1]{fontenc}
\usepackage{fancyhdr}
\usepackage[usenames,dvipsnames]{xcolor}
\usepackage{pdfcolmk}
\usepackage{/Users/SHER/Documents/Hartshorne-Exercises/style}

% Math Operators
% \DeclareMathOperator{\zz}{\mathbb{Z}} % Integers
% \DeclareMathOperator{\rr}{\mathbb{R}} % Reals
% \DeclareMathOperator{\nn}{\mathbb{N}} % Naturals
% \DeclareMathOperator{\qq}{\mathbb{Q}} % Rationals
% \DeclareMathOperator{\Ht}{\rm{ht}}    % Height of a prime ideal
% \DeclareMathOperator{\Dim}{\rm{dim}}  % Dimension of ring/space
% \DeclareMathOperator{\im}{\rm{Im}}    % Image of a map
% \let\aa\relax
% \DeclareMathOperator{\aa}{\mathbf{A}} % Affine n-space
% \DeclareMathOperator{\by}{\mathbf{y}} % Affine n-space
% \DeclareMathOperator{\fa}{\fa}
% \DeclareMathOperator{\bbP}{\mathbf{P}} % Projective space

% % Math commands
% \newcommand{\x}{x_1, \dots, x_n}    % Shortcut: "x_1, \dots, x_n" => "\x"
% \newcommand{\idl}[1]{\mathfrak{#1}} % Shortcut for ideals: "\mathfrak{p}" => "\idl{p}"
% \renewcommand{\phi}{\varphi}
% \renewcommand{\epsilon}{\varepsilon}

% % Enumerate environment will now list boldfaced letters
% \renewcommand{\labelenumi}{{\bf (\alph{enumi})}}
% \newcommand*{\ms}[1]{\ensuremath{\mathscr{#1}}}


% Header 
\newcommand{\header}[2]{
    {\noindent
    {\Large \bf Hartshorne #1 Exercises: #2}
    \hfill 
    {\large Feiyang Lin and Luke Trujillo}
    \vspace{0.5cm}}
}


\begin{document}

\header{1.2}{Projective Varieties}

% --
\begin{exercise}[Exercise 2.1]
    Prove the ``homogeneous Nullstellensatz,'' which says if $\fa
    \subseteq S$ is a homogeneous ideal, and if $f \in S$ is a homogeneous
    polynomial with $\deg f > 0$, such that $f(P) = 0$ for all $P \in
    Z(\fa)$ in $\bbP^n$, then $f^q \in \fa$ for some $q >
    0$.
\end{exercise}

\begin{solution}

\end{solution}

% --
\begin{exercise}[Exercise 2.2]
    For a homogeneous ideal $\fa \subseteq S$, show that the following
    conditions are equivalent:
    \begin{itemize}
        \item[(\emph{i}.)] $Z(\fa) = \emptyset$ (the empty set);
        \item[(\emph{ii}.)] $\sqrt{\fa} =$ either $S$ or the ideal $S_+ =
        \bigoplus_{d > 0}S_d$;
        \item[(\emph{iii}.)] $\fa \supseteq S_d$ for some $d > 0$. 
    \end{itemize}
\end{exercise}

\begin{solution}
    \lfy{
    (i) $\Rightarrow$ (ii). Since $Z(\fa) = \emptyset$, the zero set of $\fa$ in affine space is either $\emptyset$ or $\{\mathbf{0}\}.$ In the first case, we certainly have $\sqrt{\fa} = S$. In the second case, we have $\sqrt{\fa} = \{p \in S: p(0) = 0\} = S_+$.

    (ii) $\Rightarrow$ (iii). It suffices to show that there exists $d$ such that all monomials of degree $d$ lies in $\fa$. Since $S_+ \subset \sqrt{\fa},$ for each $0 \leq i \leq n$, there exists $d_i$ such that $x_i^{d_i} \in \fa$. Let $d = n\sum_{i=0}^{n}d_i$. Then if a monomial has degree $d$, then there exists $i$ such that the exponent of $x_i$ in the monomial is at least $\sum_{i=0}^{n}d_i$, and therefore at least $d_i$, which implies that the monomial is in $\fa$.

    (iii) $\Rightarrow$ (i). If $S_d \subseteq \fa$ for some $d > 0$, then $x_i \in \sqrt{\fa}$ for every $0 \leq i \leq n$, which means that $Z(\fa) = Z(\sqrt{\fa}) = \emptyset$. 
    }
\end{solution}

% --
\begin{exercise}[Exercise 2.3]
    \begin{enumerate}
        \item If $T_1 \subseteq T_2$ are subsets of $S^h$, then $Z(T_1) \supseteq
        Z(T_2)$.
      \item If $Y_1 \subseteq Y_2$ are subsets of $\bbP^n$, then $I(Y_1) \supseteq
        I(Y_2)$.
      \item For any two subsets $Y_1,Y_2$ of $\bbP^n$, $I(Y_1 \cup Y_2) = I(Y_1)
        \cap I(Y_2)$.
      \item If $\fa \subseteq S$ is a homogeneous ideal with
        $Z(\fa) \ne \emptyset$, then $I(Z(\fa)) =
        \sqrt{\fa}$.
      \item For any subset $Y \subseteq \bbP^n$, $Z(I(Y)) = \overline{Y}$.
    \end{enumerate}
\end{exercise}

\begin{solution}

\end{solution}

% --
\begin{exercise}[Exercise 2.4]
    \begin{enumerate}
        \item There is a one-to-one inclusion-reversing correspondence between
        algebraic sets in $\bbP^n$ and homogeneous radical ideals of $S$ not equal
        to $S_+$ given by $Y \mapsto I(Y)$ and $\fa \mapsto
        Z(\fa)$. \emph{Note:} Since $S_{+}$ does not occur in this correspondence, 
        it is sometimes called te \emph{irrelevant} maximal ideal of $S$. 
        
        \item An algebraic set $Y \subseteq \bbP^n$ is irreducible if and only if $I(Y)$ is a prime ideal. 
        \item Show that $\bbP^n$ itself is irreducible. 
    \end{enumerate}
\end{exercise}

\begin{solution}

\end{solution}

% --
\begin{exercise}[Exercise 2.5]
    \begin{enumerate}
        \item $\bbP^n$ is a noetherian topological space. 
        \item Every algebraic set in $\bbP^n$ can be written uniquely as a finite
          union of irreducible algebraic sets, no one containing another. These are
          called its \emph{irreducible components}. 
    \end{enumerate}
\end{exercise}

\begin{solution}

\end{solution}

\lfy{for 2.12(b)

We prove the harder direction that $Z(\fa) \subseteq \im(\rho_d).$ We may index the $N+1$ coordinates of a point in $\bbP^n$ by tuples of the form $(a_0, a_1, \dots, a_n)$ where $a_i \in \Z_{+}$ and the sum of all $a_i$'s is $d$. Given $\y \in Z(\fa)$, I claim that there exists $0 \leq i \leq n$ such that $y_{d\bbe_i} \neq 0$. Indeed, suppose towards the contrary. Then since for any index $\bv = (a_0, a_1, \dots, a_n)$, $p(\y) = y_{\bv}^d - \prod_{i = 0}^{n}y_{d \bbe_i}^{a_i} \in \fa$, we have that $p(\y) = y_{\bv}^d = 0$, which implies that $y_{\bv} = 0$ for arbitrary $\bv$, which is absurd. 

So fix some $i$ and some representative of $\y$ such that $y_{d\bbe_i} = 1$. Let $\x \in \bbP^n$ be such that $x_i = y_{d\bbe_i} = 1$ and $x_j = y_{d\bbe_i - \bbe_i + \bbe_j}$. We claim that $\rho_d(\x) = \y$. By construction, we already have that $x_i^{d-1}x_j = x_j = y_{d\bbe_i - \bbe_i + \bbe_j}$. It's straightforward to check that the polynomials in $\fa$ ensures that $y_{\bv} = M_{\bv}(\x)$.

To give an example to make this proof clearer, consider the example when $n = 1$ and $d = 3$. Then if $\y = (y_{30}, y_{21}, y_{12}, y_{03}) \in Z(\fa)$, suppose WLOG that $y_{30} = 1$. Then $\y = \rho_d(1, y_{21})$. We check that since $y_{30}y_{12} - y_{21}^2 = 0$, indeed $y_{12} = y_{21}^2$. Similarly, since $y_{03}y_{30}^2 - y_{21}^3$, indeed $y_{03} = y_{21}^3$.

2.12(c)

 The map $\rho_d$ is clearly a bijection between $\bbP^n$ and $\im \rho_d = Z(\fa)$. So it suffices to show that $\rho_d$ is bicontinuous, or equivalently, that it identifies the closed sets in $\bbP^n$ and $Z(\fa)$. 

 ($\rho_d$ continuous.) We claim that for any ideal $I \subset k[y_0, \dots, y_N]$, \[\rho_d^{-1}(Z(I)) = Z(\theta(I)).\] Notice that if $(x_0, \dots, x_n) \in \rho_d^{-1}(Z(I))$, then $p(M_0(\bx), \dots, M_N(\bx)) = 0$ for all $p(y_0, \dots, y_N) \in I$. If $(x_0, \dots, x_n) \in Z(\theta(I))$, then for all $p(y_0, \dots, y_N) \in I$, $\theta(p)(x_0, \dots, x_n) = 0$. But $\theta(p)(x_0, \dots, x_n) = p(M_0(\bx), \dots, M_N(\bx))$. So these two conditions are equivalent.

 ($\rho_d^{-1}$ continuous.) We claim that for any ideal $J \subset k[x_0, \dots, x_n]$, \[\rho_d(Z(J)) = Z(\theta^{-1}J) \cap Z(\fa).\] Indeed, if $\y = \rho_d(\x)$ where $\x \in Z(J)$, then for any $p \in \theta^{-1}J$, $p(\y) = p(\rho_d(\x)) = 0$ because $p \circ \rho_d = \theta(p) \in J$. Conversely, if $\y \in Z(\theta^{-1}J) \cap Z(\fa)$, then by part (ii), since $Z(\fa) = \im \rho_d$, there exists $\x$ such that $\y = \rho_d(\x)$. We check that $\x \in Z(J)$: for all $q$ such that $q \circ \rho_d \in J$, we have that $q(\y) = q(\rho_d(\x))= 0$. In other words, for all $p \in J$, $p(\x) = 0$.
}












\end{document}