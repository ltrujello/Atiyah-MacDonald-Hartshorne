\documentclass{hw_pset} % hw_pset simply loads article.cls and creates the exercise/solution environments.
\usepackage[margin=0.7in]{geometry}
\usepackage{graphicx}
\usepackage{amsmath, amssymb} 
\usepackage{lmodern}
\usepackage[T1]{fontenc}
\usepackage{fancyhdr}

% Math Operators
\DeclareMathOperator{\zz}{\mathbb{Z}} % Integers
\DeclareMathOperator{\rr}{\mathbb{R}} % Reals
\DeclareMathOperator{\nn}{\mathbb{N}} % Naturals
\DeclareMathOperator{\qq}{\mathbb{Q}} % Rationals
\DeclareMathOperator{\Ht}{\rm{ht}}    % Height of a prime ideal
\DeclareMathOperator{\Dim}{\rm{dim}}  % Dimension of ring/space
\DeclareMathOperator{\im}{\rm{Im}}    % Image of a map
\let\aa\relax
\DeclareMathOperator{\aa}{\mathbf{A}} % Affine n-space
\DeclareMathOperator{\pp}{\mathbf{P}} % Projective space

% Math commands
\newcommand{\x}{x_1, \dots, x_n}    % Shortcut: "x_1, \dots, x_n" => "\x"
\newcommand{\idl}[1]{\mathfrak{#1}} % Shortcut for ideals: "\mathfrak{p}" => "\idl{p}"
\renewcommand{\phi}{\varphi}
\renewcommand{\epsilon}{\varepsilon}

% Enumerate environment will now list boldfaced letters
\renewcommand{\labelenumi}{{\bf (\alph{enumi})}}
\newcommand*{\ms}[1]{\ensuremath{\mathscr{#1}}}

% Header 
\newcommand{\header}[2]{
    {\noindent
    {\Large \bf Hartshorne #1 Exercises: #2}
    \hfill 
    {\large Feiyang Lin and Luke Trujillo}
    \vspace{0.5cm}}
}


\begin{document}

\header{1.2}{Projective Varieties}

% --
\begin{exercise}[Exercise 2.1]
    Prove the ``homogeneous Nullstellensatz,'' which says if $\mathfrak{a}
    \subseteq S$ is a homogeneous ideal, and if $f \in S$ is a homogeneous
    polynomial with $\deg f > 0$, such that $f(P) = 0$ for all $P \in
    Z(\mathfrak{a})$ in $\pp^n$, then $f^q \in \mathfrak{a}$ for some $q >
    0$.
\end{exercise}

\begin{solution}
    Let $\idl{a}$ be a homogeneous ideal of $k[x_0, \dots, x_n]$. 
    Then in $\pp^n$, we have that  
    \[
        Z(\idl{a}) = \bigg\{ P \in \pp^n \;\bigg|\; \text{For all } f \in \idl{a}, f(P) = 0  \bigg\}
    \]
    while in $\aa^{n+1}$, we have that 
    \[
        Z'(\idl{a}) = \bigg\{ Q \in \aa^{n+1} \;\bigg|\; \text{For all } f \in \idl{a}, f(Q) = 0  \bigg\}.
    \]
    We can write a surjective map $\pi: Z'(\idl{a}) \to Z(\idl{a})$ sending an affine point to its projective 
    equivalence class.
    In addition, for a projective point $P \in Z(\idl{a})$, we can observe that 
    $\pi^{-1}(P) \subset Z'(\idl{a})$ (the elements of $P$'s equivalence class). 
    
    Thus, if $f$ is homogeneous with a nonzero degree and $f(P) = 0$ for all $P \in Z(\idl{a})$, 
    then $f(Q) = 0$ for all $Q \in Z'(\idl{a})$. By the usual Nullstellensatz, 
    this implies that $f^q \in \idl{a}$ for some $q > 0$, which proves the result. 
\end{solution}

% --
\begin{exercise}[Exercise 2.2]
    For a homogeneous ideal $\mathfrak{a} \subseteq S$, show that the following
    conditions are equivalent:
    \begin{itemize}
        \item[(\emph{i}.)] $Z(\mathfrak{a}) = \emptyset$ (the empty set);
        \item[(\emph{ii}.)] $\sqrt{\mathfrak{a}} =$ either $S$ or the ideal $S_+ =
        \bigoplus_{d > 0}S_d$;
        \item[(\emph{iii}.)] $\mathfrak{a} \supseteq S_d$ for some $d > 0$. 
    \end{itemize}
\end{exercise}

\begin{solution}
    \begin{description}
        \item[$(i) \implies (ii)$.] 
        Suppose $\idl{a}$ is a homogeneous ideal and $Z(\idl{a}) = \varnothing$. By the 
        homogeneous Nullstellensatz, it is vacuously true that for every 
        $f \in \oplus_{d > 0}S_d$, $f(P) = 0$ for all $P \in Z(\idl{a})$, and so 
        for every such $f$ there exists a $q > 0$ such that $f^q \in \idl{a}$. 
        Therefore, $\oplus_{d > 0}S_d \subseteq \sqrt{a}$. If $\idl{a}$ contains 
        a unit, then $\sqrt{\idl{a}} = S$. If $\idl{a}$ does not contain a unit, then 
        $\sqrt{\idl{a}} \subseteq \oplus_{d > 0}S_d \implies \oplus_{d > 0}S_d = \sqrt{\idl{a}}$.

        \item[$(ii) \implies (iii)$.] 
        Suppose $\sqrt{a} = S$. Then this implies that $1 \in \idl{a} \implies \idl{a} = S$. Hence $S_d \subset \idl{a}$.

        Alternatively, suppose $\sqrt{a} = \oplus_{d > 0}S_d$. Then for every $f \in \oplus_{d > 0}S_d$, 
        there exists a $q > 0$ such that $f^q \in \idl{a}$. In particular, there 
        exist $r_1, \dots, r_n$ such that 
        \[
            x_1^{r_1}, \dots, x_n^{r_n} \in \idl{a}.
        \]
        Take $d = r_1 + \cdots + r_n$. We claim that $S_d \subset \idl{a}$. 
        To see this, note that every degree-$d$ homogeneous polynomial is of the form 
        \[
            \sum_{k \ge 0}c_k x_1^{\alpha_1(k)} \cdots  x_n^{\alpha_n(k)}
        \]
        where only finitely many summands are nonzero and $\alpha_1(k) + \cdots + \alpha_n(k) = d$ is a sum 
        of nonnegative integers. Since 
        $d = r_1 + \cdots + r_n$, we know that for any summand $x_1^{\alpha_1(k)} \cdots  x_n^{\alpha_n(k)}$
        at least one $\alpha_i(k) \ge r_i$. Hence, the summand is in $\idl{a}$, and so the whole 
        sum is in $\idl{a}$. Therefore $S_d \subseteq \idl{a}$. 

        \item[$(iii) \implies (i)$.] 
        Suppoose $S_d \subseteq \idl{a}$ for $d > 0$. Then $Z(\idl{a})$ must 
        at least contain points $P$ such that $f(P) = 0$ for all $f \in S_d$.
        This includes the polynomials $x_1^d, \dots, x_n^d$. However, these $n$ polynomials 
        cannot all be simultaneously zero. Hence $Z(\idl{a}) = \varnothing$. 
    
    \end{description}    
\end{solution}

% --
\begin{exercise}[Exercise 2.3]
    \begin{enumerate}
        \item If $T_1 \subseteq T_2$ are subsets of $S^h$, then $Z(T_1) \supseteq
        Z(T_2)$.
      \item If $Y_1 \subseteq Y_2$ are subsets of $\pp^n$, then $I(Y_1) \supseteq
        I(Y_2)$.
      \item For any two subsets $Y_1,Y_2$ of $\pp^n$, $I(Y_1 \cup Y_2) = I(Y_1)
        \cap I(Y_2)$.
      \item If $\mathfrak{a} \subseteq S$ is a homogeneous ideal with
        $Z(\mathfrak{a}) \ne \emptyset$, then $I(Z(\mathfrak{a})) =
        \sqrt{\mathfrak{a}}$.
      \item For any subset $Y \subseteq \pp^n$, $Z(I(Y)) = \overline{Y}$.
    \end{enumerate}
\end{exercise}

\begin{solution}
    I did {\bf (a) -- (c)}, but I don't feel like \TeX-ing them.
    \begin{enumerate}
        \setcounter{enumi}{3}
        \item Suppose $f$ is homogeneous and $f \in I(Z(\idl{a}))$. Then 
        $f(P) = 0$ for all $P \in Z(\idl{a})$. By the homogeneous Nullstellensatz, 
        we see that $f \in \sqrt{\idl{a}}$.
        
        Suppose $f \in \idl{a}$. Then $f^q \in \idl{a}$ for some $q > 0$. 
        As $f^q(P) = 0$ for all $P \in Z(\idl{a})$, $f(P) = 0$ for all 
        $P \in Z(\idl{a})$ (as $k$ is an integral domain) and so it 
        follows that $f \in I(Z(\idl{a}))$.
        Our total work then shows that $I(Z(\idl{a})) = \sqrt{\idl{a}}$, as desired.

        \item
        We first know that $Y \subset Z(I(Y))$; we show that it is the smallest closed 
        set containing $Y$. Thus
        let $Z(J)$ be a closed set where $Y \subset Z(J) \subset I(Z(Y))$. 
        Then $I(Z(J)) \subset I(Y)$. Since $I(Z(J)) = \sqrt{J}$, we see that 
        $J \subset I(Y)$. Therefore, 
        $Z(I(Y)) \subset Z(J)$. Hence, $Z(I(Y)) = Z(J)$, and so $Z(I(Y)) = \overline{Y}$.  

        
    
    \end{enumerate}
\end{solution}

% --
\begin{exercise}[Exercise 2.4]
    \begin{enumerate}
        \item There is a one-to-one inclusion-reversing correspondence between
        algebraic sets in $\pp^n$ and homogeneous radical ideals of $S$ not equal
        to $S_+$ given by $Y \mapsto I(Y)$ and $\mathfrak{a} \mapsto
        Z(\mathfrak{a})$. \emph{Note:} Since $S_{+}$ does not occur in this correspondence, 
        it is sometimes called te \emph{irrelevant} maximal ideal of $S$. 
        
        \item An algebraic set $Y \subseteq \pp^n$ is irreducible if and only if $I(Y)$ is a prime ideal. 
        \item Show that $\pp^n$ itself is irreducible. 
    \end{enumerate}
\end{exercise}

\begin{solution}

\end{solution}

% --
\begin{exercise}[Exercise 2.5]
    \begin{enumerate}
        \item $\pp^n$ is a noetherian topological space. 
        \item Every algebraic set in $\pp^n$ can be written uniquely as a finite
          union of irreducible algebraic sets, no one containing another. These are
          called its \emph{irreducible components}. 
    \end{enumerate}
\end{exercise}

\begin{solution}

\end{solution}

\begin{exercise}[2.6]
    If $Y$ is a projective variety with homogeneous coordinate ring $S(Y)$, show that $\dim S(Y) = \dim Y + 1$. 
    [\emph{Hint}: Let $\phi_i: U_i \to \aa^n$ be the homeomorphism of (2.2), 
    let $Y_i$ be the affine variety $\phi_i(Y \cap U_i)$ and let $A(Y_i)$ be its affine coordinate ring.
    Show that $A(Y_i)$ can be identified with the subring of elements of degree 0 of the 
    localized ring $S(Y)x_i$. Then show that $S(Y)_{x_i} \cong A(Y_i)[x_i, x_i^{-1}]$. 
    Now use (1.7), (1.8A), and (Ex 1.10), and look at the transcendence degrees. Conclude also that 
    $\Dim Y = \Dim Y_i$ whenever $Y_i$ is empty.]
\end{exercise}

\begin{solution}

\end{solution}

\begin{exercise}[2.7]
    \begin{enumerate}
        \item $\dim \pp^n = n$.
        \item If $Y \subseteq \pp^n$ is a quasi-projective variety, then $\dim Y =
          \dim \overline{Y}$.
    \end{enumerate}
\end{exercise}

\begin{solution}
    
\end{solution}

\begin{exercise}[2.8]
    A projective variety $Y \subseteq \pp^n$ has dimension $n-1$ if
    and only if it is the zero set of a single irreducible homogeneous polynomial
    $f$ of positive degree. $Y$ is called a \emph{hypersurface} in $\pp^n$.   
\end{exercise}

\begin{solution}
    
\end{solution}

\begin{exercise}[2.9]
    If $Y \subseteq \aa^n$ is an affine variety, we identify $\aa^n$ with an open
    set $U_0 \subset \pp^n$ by the homeomorphism $\varphi_0$. Then we can speak of
    $\overline{Y}$, the closure of $Y$ in $\pp^n$, which is called the
    \emph{projective closure} of $Y$. 
\end{exercise}

\begin{solution}
    
\end{solution}

\begin{exercise}[2.10]
    Let $Y \subset \pp^n$ be a nonempty algebraic set, and let $\theta\colon
    \aa^{n+1} \setminus \{(0,\cdots,0\} \to \pp^n$ be the map which sends the
    point with affine coordinates $(a_0, \cdots, a_n)$ to the point with
    homogeneous coordinates $[a_0 :\cdots: a_n]$. We define the \emph{affine cone}
    over $Y$ to be $$C(Y) = \theta^{-1}(Y) \cup \{(0,\cdots, 0)\}.$$ 
    \begin{enumerate}
        \item Show that $C(Y)$ is an algebraic set in $\aa^{n+1}$, whose ideal is equal to $I(Y)$, considered as an ordinary ideal in $k[x_0, \cdots, x_n]$. 
        \item $C(Y)$ is irreducible if and only if $Y$ is irreducible. 
        \item $\dim C(Y) = \dim Y +1$.
    \end{enumerate}
    Sometimes we consider the projective closure $\overline{C(Y)}$ of $C(Y)$ in
    $\pp^{n+1}$. This is called the \emph{projective cone} over $Y$.
\end{exercise}

\begin{solution}
    
\end{solution}

\begin{exercise}[2.11]
    A hypersurface defined by a linear polynomial is called a \emph{hyperplane}. 
        \begin{enumerate}
            \item Show that the following two conditions are equivalent for a variety $Y \subset \pp^n$: 
                \begin{itemize}
                \item[(\emph{i}.)] $I(Y)$ can be generated by linear polynomials
                \item[(\emph{ii}.)] $Y$ can be written as an intersection of hyperplanes. 
                \end{itemize}
                In this case we say that $Y$ is a \emph{linear variety} in $\pp^n$.
            \item If $Y$ is a linear variety of dimension $r$ in $\pp^n$, show that $I(Y)$ is minimally generated by $n-r$ linear polynomials 
            \item Let $Y,Z$ be linear varieties in $\pp^n$, with $\dim Y = r$, $\dim Z = s$. If $r+s-n \geq 0$, then $Y \cap Z \neq \emptyset$. Furthermore, if $Y \cap Z \neq \emptyset$, then $Y \cap Z$ is a linear variety of dimension $\geq r+s-n$ (Think of $\AA^{n+1}$ as a vector space over $k$, and work with its subspaces.)
        \end{enumerate}
\end{exercise}

\begin{solution}
    
\end{solution}

\begin{exercise}[2.12]
    For given $n, d>0$, let
    $M_0,\ldots, M_N$ be all the monomials of degree $d$ in the $n+1$ variables
    $x_0, \ldots x_n$, where $N = \binom{n+d}{n} -1.$ We define a mapping
    $\rho_d\colon \pp^n \to \pp^N$ by sending the point $P = (a_0, \ldots, a_n)$
    to the point $\rho_d(P) = (M_0(a), \ldots, M_N(a))$ obtained by substituting
    the $a_i$ in the monomials $M_j$. This is called the $d$-uple \emph{embedding}
    of $\pp^n$ in $\pp^N$. For example, if $n=1, d=2$, then $N= 2$, and the image
    $Y$ of the $2$-uple embedding of $\pp^1$ in $\pp^2$ is a conic. 
    \begin{enumerate}
      \item Let $\theta\colon k[y_0, \ldots, y_N] \to k[x_0, \ldots, x_n]$ be the
        homorphism defined by sending $y_i$ to $M_i$, and let $\mathfrak{a}$ be
        the kernel of $\theta$. Then $\mathfrak{a}$ is homogeneous prime ideal, and
        so $Z(\mathfrak{a})$ is a projective variety in $\pp^N$. 
      \item Show that the image of $\rho_d$ is exactly $Z(\mathfrak{a})$.
      \item Now show that $\rho_d$ is a homeomorphism of $\pp^n$ onto the projective
        variety $Z(\mathfrak{a})$. 
      \item Show that the twisted cubic curve in $\pp^3$
        {\emph{(Ex.\ $2.9$)}} is equal to the $3$-uple
        embedding of $\pp^1$ in $\pp^3$, for suitable choice of coordinates. 
    \end{enumerate}
\end{exercise}

\begin{solution}
    
\end{solution}

\begin{exercise}[2.13]
    Let $Y$ be the image of the $2$-uple embedding of $\pp^2$ in $\pp^5$. This is the
    \emph{Veronese surface}. If $Z \subseteq Y$ is a closed curve (a \emph{curve} is 
    a variety of dimension $1$), show that there exists a hypersurface $V \subseteq
    \pp^5$ such that $V \cap Y = Z$. 
\end{exercise}

\begin{solution}
    
\end{solution}

\begin{exercise}[2.14]
    Let $\psi\colon \pp^r \times \pp^s \to \pp^N$ be the map defined by sending
    the ordered pair $(a_0, \ldots, a_r) \times (b_0, \ldots, b_s)$ to $(\ldots,
    a_ib_j, \ldots)$ in lexicographic order, where $N = rs + r +s$. Note that
    $\psi$ is well-defined and injective. It is called the \emph{Segre embedding}.
    Show that the image of $\psi$ is a subvariety of $\pp^N$.
\end{exercise}

\begin{solution}
    
\end{solution}

\begin{exercise}[2.15]
    Consider the surface $Q$ (a \emph{surface} is variety of dimension $2$) in
    $\pp^3$ defined by the equation $xy-zw = 0$.
    \begin{enumerate} 
      \item Show that $Q$ is equal to the Segre embedding of $\pp^1 \times \pp^1$ in
        $\pp^3$, for suitable choice of coordinates.
      \item Show that $Q$ contains two families of lines (a \emph{line} is a
        linear variety of dimension $1$) $\{L_t\}, \{M_t\}$, each parametrized by
        $t \in \pp^1$, with the properties that if $L_t \ne L_u$, then
        $L_t \cap L_u = \emptyset$; if $M_t \ne M_u$, $M_t \cap M_u = \emptyset$,
        and for all $t,u, L_t \cap M_u =$ one point. 
      \item Show that $Q$ contains other curves besides these lines, and deduce that
        the Zariski topology on $Q$ is not homeomorphic via $\psi$ to the product
        topology on $\pp^1 \times \pp^1$ (where each $\pp^1$ has its Zariski
        topology). 
    \end{enumerate}
\end{exercise}

\begin{solution}
    
\end{solution}

\begin{exercise}[2.16]
    \begin{enumerate}
        \item The intersection of two varieties need not be a variety. For example,
        let $Q_1$ and $Q_2$ be the quadric surfaces in $\pp^3$ given by the
        equations $x^2-yw = 0$ and $xy - zw =0$, respectively. Show that $Q_1 \cap
        Q_2$ is the union of a twisted cubic curve and a line.
        
        \item Even if the intersection of two varieties is a variety, the ideal of the
        intersection may not be the sum of the ideals. For example, let $C$ be the
        conic in $\pp^2$ given by the equation $x^2-yz = 0$. Let $L$ be the line
        given by $y = 0$. Show that $C \cap L$ consists of one point $P$, but that
        $I(C) + I(L) \ne I(P)$. 
    \end{enumerate}
\end{exercise}

\begin{solution}
    
\end{solution}

\begin{exercise}[2.17]
    A variety $Y$ of dimension $r$ in $\pp^n$ is a \emph{(strict) complete
    intersection} if $I(Y)$ can be generated by $n-r$ elements. $Y$ is a
    \emph{set-theoretic complete intersection} if $Y$ can be written as the
    intersection of $n-r$ hypersurfaces.
    \begin{enumerate}
    \item Let $Y$ be a variety in $\pp^n$, let $Y = Z(\mathfrak{a})$; and
    suppose that $\mathfrak{a}$ can be generated by $q$ elements. Then show
    that $\dim Y \ge n-q$. 
    \item Show that a strict complete intersection is a set-theoretic complete
    intersection.
    \item The converse of $(b)$ is false. For example let $Y$ be the twisted
    cubic curve in $\pp^3$ {\emph{(Ex.\ 2.9)}}. Show that
    $I(Y)$ cannot be generated by two elements. On the other hand, find
    hypersurfaces $H_1,H_2$ of degrees $2,3$ respectively, such that
    $Y = H_1 \cap H_2$. 
    \item It is an unsolved problem whether every closed irreducible curve in
    $\pp^3$ is a set-theoretic intersection of two surfaces.
    \end{enumerate}
\end{exercise}

\begin{solution}
    
\end{solution}

\end{document}