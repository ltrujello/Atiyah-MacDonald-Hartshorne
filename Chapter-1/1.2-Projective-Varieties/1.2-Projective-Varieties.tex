\documentclass{/Users/SHER/Documents/Hartshorne-Exercises/hw_pset} % hw_pset simply loads article.cls and creates the exercise/solution environments.
% \documentclass{hw_pset}
\usepackage[margin=0.7in]{geometry}
\usepackage{graphicx}
\usepackage{amsmath, amssymb} 
\usepackage{lmodern}
\usepackage[T1]{fontenc}
\usepackage{fancyhdr}

% Math Operators
\DeclareMathOperator{\zz}{\mathbb{Z}} % Integers
\DeclareMathOperator{\rr}{\mathbb{R}} % Reals
\DeclareMathOperator{\nn}{\mathbb{N}} % Naturals
\DeclareMathOperator{\qq}{\mathbb{Q}} % Rationals
\DeclareMathOperator{\Ht}{\rm{ht}}    % Height of a prime ideal
\DeclareMathOperator{\Dim}{\rm{dim}}  % Dimension of ring/space
\DeclareMathOperator{\im}{\rm{Im}}    % Image of a map
\let\aa\relax
\DeclareMathOperator{\aa}{\mathbf{A}} % Affine n-space
\DeclareMathOperator{\pp}{\mathbf{P}} % Projective space

% Math commands
\newcommand{\x}{x_1, \dots, x_n}    % Shortcut: "x_1, \dots, x_n" => "\x"
\newcommand{\idl}[1]{\mathfrak{#1}} % Shortcut for ideals: "\mathfrak{p}" => "\idl{p}"
\renewcommand{\phi}{\varphi}
\renewcommand{\epsilon}{\varepsilon}

% Enumerate environment will now list boldfaced letters
\renewcommand{\labelenumi}{{\bf (\alph{enumi})}}
\newcommand*{\ms}[1]{\ensuremath{\mathscr{#1}}}

% Header 
\newcommand{\header}[2]{
    {\noindent
    {\Large \bf Hartshorne #1 Exercises: #2}
    \hfill 
    {\large Feiyang Lin and Luke Trujillo}
    \vspace{0.5cm}}
}


\begin{document}

\header{1.2}{Projective Varieties}

% --
\begin{exercise}[Exercise 2.1]
    Prove the ``homogeneous Nullstellensatz,'' which says if $\mathfrak{a}
    \subseteq S$ is a homogeneous ideal, and if $f \in S$ is a homogeneous
    polynomial with $\deg f > 0$, such that $f(P) = 0$ for all $P \in
    Z(\mathfrak{a})$ in $\pp^n$, then $f^q \in \mathfrak{a}$ for some $q >
    0$.
\end{exercise}

\begin{solution}

\end{solution}

% --
\begin{exercise}[Exercise 2.2]
    For a homogeneous ideal $\mathfrak{a} \subseteq S$, show that the following
    conditions are equivalent:
    \begin{itemize}
        \item[(\emph{i}.)] $Z(\mathfrak{a}) = \emptyset$ (the empty set);
        \item[(\emph{ii}.)] $\sqrt{\mathfrak{a}} =$ either $S$ or the ideal $S_+ =
        \bigoplus_{d > 0}S_d$;
        \item[(\emph{iii}.)] $\mathfrak{a} \supseteq S_d$ for some $d > 0$. 
    \end{itemize}
\end{exercise}

\begin{solution}
    
\end{solution}

% --
\begin{exercise}[Exercise 2.3]
    \begin{enumerate}
        \item If $T_1 \subseteq T_2$ are subsets of $S^h$, then $Z(T_1) \supseteq
        Z(T_2)$.
      \item If $Y_1 \subseteq Y_2$ are subsets of $\pp^n$, then $I(Y_1) \supseteq
        I(Y_2)$.
      \item For any two subsets $Y_1,Y_2$ of $\pp^n$, $I(Y_1 \cup Y_2) = I(Y_1)
        \cap I(Y_2)$.
      \item If $\mathfrak{a} \subseteq S$ is a homogeneous ideal with
        $Z(\mathfrak{a}) \ne \emptyset$, then $I(Z(\mathfrak{a})) =
        \sqrt{\mathfrak{a}}$.
      \item For any subset $Y \subseteq \pp^n$, $Z(I(Y)) = \overline{Y}$.
    \end{enumerate}
\end{exercise}

\begin{solution}

\end{solution}

% --
\begin{exercise}[Exercise 2.4]
    \begin{enumerate}
        \item There is a one-to-one inclusion-reversing correspondence between
        algebraic sets in $\pp^n$ and homogeneous radical ideals of $S$ not equal
        to $S_+$ given by $Y \mapsto I(Y)$ and $\mathfrak{a} \mapsto
        Z(\mathfrak{a})$. \emph{Note:} Since $S_{+}$ does not occur in this correspondence, 
        it is sometimes called te \emph{irrelevant} maximal ideal of $S$. 
        
        \item An algebraic set $Y \subseteq \pp^n$ is irreducible if and only if $I(Y)$ is a prime ideal. 
        \item Show that $\pp^n$ itself is irreducible. 
    \end{enumerate}
\end{exercise}

\begin{solution}

\end{solution}

% --
\begin{exercise}[Exercise 2.5]
    \begin{enumerate}
        \item $\pp^n$ is a noetherian topological space. 
        \item Every algebraic set in $\pp^n$ can be written uniquely as a finite
          union of irreducible algebraic sets, no one containing another. These are
          called its \emph{irreducible components}. 
    \end{enumerate}
\end{exercise}

\begin{solution}

\end{solution}














\end{document}