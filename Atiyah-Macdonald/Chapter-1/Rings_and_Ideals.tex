\documentclass[10pt]{amsart}
\usepackage[margin=0.7in]{geometry}
\usepackage{graphicx}
\usepackage{amsmath, amssymb, amsthm} 
\usepackage{lmodern}
\usepackage[T1]{fontenc}
\usepackage{fancyhdr}
\usepackage[usenames,dvipsnames]{xcolor} % If using tikz, put \usepackage{tikz} after this
\usepackage{pdfcolmk}

% Our commands + environments are in style.cls. 
\usepackage{../atiyah-macdonald-style}

\begin{document}
\header{Chapter 1}

\begin{exercise}
    Let $x$ be a nilpotent element of a ring $A$. Show that $1 + x$ is a unit of $A$. Deduce that the sum of a nilpotent element and a unit is a unit.
\end{exercise}

\begin{solution}
    Let $x$ be nilpotent and $n$ the smallest integer such that $x^n = 0$. The case for $n = 2$ is clear 
    because $(1 + x)(1 - x) = 1 - x^2 = 1$. Therefore, let $n > 2$, and let $k$ be the smallest integer 
    such that $n \le 2^k$. Then observe that 
    \[
        (1 + x)(1 - x) \prod_{i = 2}^{k}\left(1 + x^{2^i}\right) = 1.
    \]
    Hence we see that $1 + x$ is a unit. By adjusting the above argument, and noting that the 
    set of units of a ring form a group, we can get that the sum of a nilpotent element and a 
    unit is a unit. 
\end{solution}


\begin{exercise}
    Let $A$ be a ring and let $A[x]$ be the ring of polynomials in an indeterminate $x$, with coefficients in $A$. Let $f = a_0 + a_1x + \dots + a_nx^n$ in $A[x]$. 
    Prove that
    \begin{itemize}
        \item[\emph{i})] $f$ is a unit in $A[x] \iff a_0$ is a unit in $A$ and $a_1, \dots, a_n$ are nilpotent. [If $b_0 + b_1x +\dots+ b_mx^m$ 
        is the inverse of $f$, prove by induction on $r$ that $a_n^{r+1}b_{m-r} = 0$. Hence show that $a_n$ is nilpotent, and then use Ex. 1.]
        \item[\emph{ii})] $f$ is nilpotent $\iff a_0, a_1, \dots a_n$ are nilpotent.
        \item[\emph{iii})] $f$ is a zero-divisor $\iff$ there exists a $a \ne 0$ in $A$ such that $af = 0.$ [Choose a
        polynomial $g = b_0 +b_1x + \dots + b_mx^m$ of least degree $m$ such that $fg = 0$. Then $a_nb_m = 0$, hence $a_ng = 0$ 
        (because $a_n g$ annihilates $f
        $ and has degree $<m$). Now show by induction that $a_{n-r} g = 0$ $(0 \le r \le n)$.]
        \item[\emph{iv})] $f$ is said to be primitive if $(a_0, a_1, \dots , a_n) = (1)$. Prove that if $f,g \in A[x]$, then
        $fg$ is primitive $\iff$ $f$ and $g$ are primitive.
    \end{itemize}
\end{exercise}

\begin{solution}
    \begin{itemize}
        \item[\emph{i})] Suppose that $f$ is a unit. Then there exists a polynomial $g$ such that 
        \[
            fg = 1 \implies \sum_{k = 0}^{n + m}\left( \sum_{i + j = k} a_ib_j \right)x^k = 1.
        \] 
        For this to be the case, we must have that $a_0b_0 = 1$, which implies that $a_0$ is a unit, 
        and $a_nb_m = 0$, which implies that $a_n$ is a zero divisor. We'll now argue that $a_n^{r+1}b_{m-r} = 0$ whenever 
        $r > 0$ (here we treat $b_{m-r} = 0$ whenever $r > m$.)
        Since the base case is true, suppose that it is true for all integers less than $r \ge 0$. 
        Then observe that the coefficient of $x^{n+m-r}$ is given by 
        \[
            \sum_{i + j = m+n-r} a_ib_j = a_nb_{m-r} + a_{n-1}b_{m-r-1} 
            + \cdots +
            a_{n-r}b_{m}
            =
            0.
        \]
        If we multiply this by $a_n^{r}$, we see that
        \[
            a_n^r \cdot a_nb_{m-r} + a_n^r \cdot a_{n-1}b_{m-r-1} 
            + \cdots +
            a_n^r \cdot a_{n-r}b_{m}
            =
            a_n^{r+1}b_{m-r}
        \] 
        where for every summand other than the first term we applied the induction hypothesis. 
        This then implies that $a_n^{r+1}b_{m-r} = 0$. Hence, we have proved our claim by 
        induction.
        Using our claim, note that $a_n^{m+1}b_0 = 0$. Since $b_0$ is a unit, we have that 
        $a_n^{m+1}=0$ which implies $a_n$ is nilpotent. 

        \item[\emph{ii})] 
        We can prove the forward statement by induction on degree. First note that 
        the base case of $\deg f = 0$ is immediate. Therefore, let $f$ have degree $n > 0$
        and suppose the (forward) statement is true for polynomials of degree $n-1$. 
        Since $f$ is nilpotent, we have that $f^r = 0$ for some integer $r$. In particular, we have that 
        $a_n^rx^{n+r} = 0 \implies a_n^r =0$. This shows that $a_n$ is nilpotent. Now observe that 
        \[
            f - a_nx^n = a_0 + \cdots a_{n-1}x^{n-1}.
        \]
        The left hand side is nilpotent, and so $a_0 + \cdots + a_{n-1}x^{n-1}$ must be nilpotent. 
        We can then apply our induction hypothesis to conclude that $a_0, \dots, a_{n-1}$ must be nilpotent. 
        This proves the forward direction. 

        To prove the reverse direction, suppose $a_i$ are nilpotent. Then
        $f^r$ is a polynomial whose coefficients are (up to a scalar multiple)
        of the form 
        \[
            a_0^{k_0}a_1^{k_1}\cdots a_n^{k_n}
        \]
        where $k_i$ are nonnegative and sum to $r$. Since each $a_i$ is nilpotent, take 
        $r = \ord(a_0) + \cdots + \ord(a_n)$. Then each coefficient $a_0^{k_0}a_1^{k_1}\cdots a_n^{k_n}$ 
        of $f^r$ must be zero, since at for at least one $i$, we have that $k_i \ge \ord(a_i)$ (or else 
        the $k_i$ powers cannot sum to $r$). In taking $r$ to be this value, we can see 
        that each coefficient of $f^r$ is zero, which implies that $f$ is nilpotent. This proves 
        the reverse direction, and completes the if and only if proof.

        \item[\emph{iii})] We follow the hint which proves the base case of our induction: 
        as $g$ is supposed to have minimal degree for which $fg = 0$, we can only have that 
        $a_ng = 0$. To prove further that $a_{n-r}g = 0$, we suppose that the statement is 
        true for all nonnegative integers less than $r$. Observe that the coefficient of 
        $x^{m+n-r}$ is given by 
        \[
            \sum_{i + j = m+n-r} a_ib_j = a_nb_{m-r} + a_{n-1}b_{m-r-1} 
            + \cdots +
            a_{n-(r+1)}b_{m+1}
            +
            a_{n-r}b_{m}
            =
            0.
        \]
        By our induction hypothesis, each term $a_nb_{m-r}, \dots, a_{n-(r+1)}b_{m+1}$ 
        must be zero since $a_{n}g, \dots, a_{n-(r+1)}g$ are all zero. This leaves 
        just $a_{n-r}b_m = 0$, from which we can again deduce that $a_{n-r}g = 0$ as it has degree less 
        than $m$ and annihilates $f$. Hence, we have that the statement is true for all nonnegative 
        integers. 

        Using this claim, we can then prove the main result by picking a nonzero 
        coefficient of $g$. Take for instance $b_m$: Then $b_mf = b_ma_0 + \cdots + b_ma_nx^n = 0$.
        This proves the main result.

        \item[\emph{iv})]
        If $fg$ is primitive, then this implies that the ideal generated by the coefficients of $fg$ 
        is the entire ring. Explicitly, there exist coefficients $c_0, \dots, c_{n+m}$ 
        such that 
        \[
            c_0\left( \sum_{i + j = 0}a_ib_j \right) 
            +
            c_1\left( \sum_{i + j = 1}a_ib_j \right)  
            + \cdots + 
            c_{n+m}\left( \sum_{i + j = n+m}a_ib_j \right)
            = 1
        \]
        Since the above is a linear relation, we can rearrange the 
        coefficients to obtain a summation of coefficients in $a_i$ to 1. 
        This can similarly be done for $b_i$. Hence $(a_0, \dots, a_n) = (1)$ 
        and $(b_0, \dots, b_{m}) = (1)$.

    \end{itemize}
\end{solution}

\begin{exercise}
    Generalize the results of Exercise 2 to a polynomial ring $A[x_1, \dots, x_r]$ in several indeterminates.
\end{exercise}

\begin{solution}
    \begin{itemize}
        \item[\emph{i})] We claim that a multivariate polynomial $f(x_1, \dots, x_n) = \sum a(i_1, \dots, i_n)x_1^{i_1}\cdots x_n^{i_n}$ 
        is a unit if and only if $a(0,\dots, 0)$ is a unit and the rest of $a(i_1, \dots, i_n)$ are nilpotent.
        We can prove this by induction on the number of variables: 
        Suppose the statement is true for polynomials in $n-1$ variables. Then any polynomial 
        $f(x_1, \dots, x_n)$ in $n$-variables is technically a polynomial in $A[x_1, \dots, x_{n-1}][x_n]$. 
        By Exercise 2.1, this holds if and only if each $a(i_1, \dots, i_n)x_1^{i_1}\cdots x_n^{i_{n-1}}$
        is nilpotent with $a(0,\dots,0)$ a unit. 
        However, this occurs if and only if the 
        $a(i_1, \dots, i_n)$ (of course excluding $a(0, \dots, 0)$) 
        are all nilpotent, which proves the result.

        \item[\emph{ii})] We claim that a multivariate polynomial is nilpotent if and only if each of 
        its coefficients are nilpotent in $A$. This is achieved by induction on the number of indeterminates 
        as in the previous example exercise

        \item[\emph{iii})] We can prove this similarly to Exercise 1.2.3. 
        First, we define the degree of a multivariate monomial $x_1^{\alpha_1} \cdots x_n^{\alpha_n}$ 
        to be the sum of the degrees of each $\alpha_i$. We denote this degree by $(\alpha_1, \dots, \alpha_n)$. 
        We then impose a total ordering on monomial 
        degree via lexicographical ordering. In other words, we say
        $x_1^{\alpha_1} \cdots x_n^{\alpha_n}$ has less degree than $x_1^{\beta_1} \cdots x_n^{\beta_n}$ 
        if and only if $\alpha_i \le \beta_i$ for all $i$. 

        With that said, let $f(x_1, \dots, x_n)$ be a zero divisor with degree $(k_1, \dots, k_n)$, 
        and let $g(x_1, \dots, x_n)$ be the polynomial with least degree $(m_1, \dots, m_n)$. 
        If we express 
        \[
            f = \sum_{}a_{i_1, \dots, i_n}x_1^{i_1}\cdots x_n^{i_n}
            \;\;
            g = \sum_{}b_{i_1, \dots, i_n}x_1^{i_1}\cdots x_n^{i_n}
        \]
        Then we see that $a(k_1, \dots, k_n)\cdot b_{m_1, \dots, m_n} = 0$. 
        Hence $a(k_1, \dots, k_n)\cdot g$ annihilates $f$, but has less degree than 
        $g$, and so it must be zero. 

        We show by induction that $a_{i_1, \dots, i_n}g = 0$ for each coefficient $a_{i_1, \dots, i_n}$ 
        of $f$. To see this, suppose $f$ has $j$-many monomial terms; then order them and denote them 
        as $a_0, \dots, a_j$; similarly suppose $g$ has $\ell$-many monomial terms and order them as $b_0, \dots, b_{\ell}$. 
        To perform induction, suppose that $a_{j - r}g = 0$ for all $0 \le r < t$ where $0 \le t < j$ (note the base case is true).
        Then observe that 
        \begin{align*}
            fh = 0 
            &\implies (a_1x_1^{i^{(1)}_1} \cdots x_n^{i^{(1)}_n} + \cdots + a_jx_1^{i^{(j)}_1} \cdots x_n^{i^{(j)}_n})h = 0 \\
            &\implies (a_1x_1^{i^{(1)}_1} \cdots x_n^{i^{(1)}_n} + \cdots + a_kx_1^{i^{(k)}_1} \cdots x_n^{i^{(k)}_n})h = 0 \\
            &\implies a_kb_{\ell} = 0 \\
            &\implies a_kh = 0.
        \end{align*}
        This completes our inductive step. We then take any nonzero coefficient of $h$ and can now use 
        our above result to conclude that $f$ may be annihilated by a single element of $A$, as desired. 

        





    \end{itemize}
\end{solution}

\begin{exercise}
    In the ring $A[x]$, the Jacobson radical is equal to the nilradical.
\end{exercise}

\begin{solution}
    It is clear that the nilradical is contained in the Jacobson radical. 
    Therefore, let $f$ be in the Jacobson radical. Then $1 - fg$ is a unit for 
    all $g \in A[x]$. In particular, $1 - f\cdot x$ is a unit. By Ex 1.2.1, 
    this implies that the coefficients of $f$ are nilpotent. By Ex 1.2.2, 
    this implies that $f$ is nilradical, so the Jacobson is contained in the nilradical. 
    This then implies that they are equal.
\end{solution}

\begin{exercise}
    Let $A$ be a ring and let $A[[x]]$ be the ring of formal power series $f = \sum_{n = 0}^{\infty}a_nx^n$ with coefficients in $A$. Show that
    \begin{itemize}
        \item[\emph{i})] $f$ is a unit in $A[[x]] \iff a_0$ is a unit in $A$.
        \item[\emph{ii})] If $f$ is nilpotent, then $a_n$ is nilpotent for all $n \ge 0$. Is the converse true? (See Chapter 7, Exercise 2.)
        \item[\emph{iii})] $f$ belongs to the Jacobson radical of $A[[x]] \iff a_0$ belongs to the Jacobson radical of A.
        \item[\emph{iv})] The contraction of a maximal ideal $\mathfrak{m}$ of $A[[x]]$ is a maximal ideal of $A$, and
        $\idl{m}$ is generated by $\idl{m}^c$ and $x$.
        \item[\emph{v})] Every prime ideal of $A$ is the contraction of a prime ideal of $A[[x]]$. 
    \end{itemize}
\end{exercise}

\begin{solution}
    \begin{itemize}
        \item[\emph{i})] Suppose $f = \sum_{n=0}^{\infty} a_nx^n$ is a power series and 
        $a_0$ is a unit. We prove that there exists an inverse of $g= \sum_{n=0}^{\infty}b_n x^n$ 
        by inductively constructing $b_n$.

        For $n=0$, we let $b_0$ be the inverse of $a_0$. Thus suppose that the coefficients have been 
        defined for all nonnegative integers less than $n > 0$. Then observe that 
        we require 
        \[
            \sum_{i=0}^n a_ib_{n-i} = 0.
        \]
        The above is a linear equation which is in terms of $b_0, \dots, b_n$.
        We can use our induction hypothesis to solve for $b_n$ and construct it in this way. We see 
        by induction that we may construct an inverse $g$, demonstrating that $f$ is a unit. 
    
        \item[\emph{ii})] Suppose $f$ is nilpotent. Then clearly $a_0$ is nilpotent. 
        Suppose now that $a_r$ is nilpotent for all $0 \le r < k$ where $0 < k < n$. 
        To show that $a_k$ is nilpotent, note that 
        \[
            f - (a_0 + a_1x + \cdots + a_{r}x^r) = \sum_{i = n}^{\infty}a_ix^i 
        \] 
        is nilpotent. Hence, $\sum_{i = n}^{\infty}a_ix^i  = x^k\sum_{i = n}^{\infty}a_ix^{i - k}$
        is nilpotent, which implies that $\sum_{i = n}^{\infty}a_ix^{i - k}$ is nilpotent. In this case, 
        we clearly have that $a_k$ is nilpotent, which proves our inductive step. We thus have the general 
        claim by induction. 

        \item[\emph{iii})] We know that $f$ is in the Jacobson $\iff$ 
        $1 - fg$ is a unit for all $g \in A[[x]]$. However, this is the case if and 
        only if $1 - a_0y$ is a unit 
        for all $y \in A$, so that $a_0$ is in the Jacobson if and ony if $f$ is in the Jacobson of $A[[x]]$.

          
          
         
    \end{itemize}
\end{solution}

\begin{exercise}
    A ring $A$ is such that every ideal not contained in the nilradical contains a non-zero idempotent (that is, an element $e$ 
    such that $e^2 = e  \ne 0$). Prove that the nilradical and Jacobson radical of $A$ are equal.
\end{exercise}

\begin{solution}
    We already know that the nilradical is contained in the Jacobson radical. 
    To show the opposite inclusion, suppose the contrary. Then since the Jacobson 
    is not contained in the nilradical, there exists an element $e$ in the Jacobson such 
    that $e^2 = e \ne 0$. As an element of the Jacobson we know that 
    $1 - e$ is a unit. However, note that $(1 - e)(1 + e) = 1 - e^2 = 1 - e$. 
    Thus $(1 - e)(1 + e - 1) = 0 \implies e(1 - e) = 0$. However, $e \ne 0$ and 
    $1 - e$ is not a zero divisor, so this is a contradiction. Hence the Jacobson is 
    contained in the nilradical, so that they are equal. 
\end{solution}

\begin{exercise}
    Let $A$ be a ring in which every element $x$ satisfies $x^n = x$ for some $n > 1$ (depending on $x$). Show that every prime ideal in $A$ 
    is maximal.
\end{exercise}

\begin{solution}
    Let $P$ be a prime ideal, and consider the ring $A/P$. Consider an element 
    $a + P$ in $A/P$. Then $(a + P)^n = a + P$ for some $n$. In particular, we have that 
    $(a + P)((a + P)^{n-1} - (1 + P)) = 0$. Since $A/P$ is an integral domain, this implies that 
    $(a + P)^{n-1} = 1 + P$. Hence, $(a + P)$ is invertible. Since this was an arbitray element 
    this implies that $A/P$ is a field, so that $P$ is maximal.
\end{solution}

\begin{exercise}
    Let $A$ be a ring $\ne 0$. Show that the set of prime ideals of $A$ has minimal 
    elements with respect to inclusion.
\end{exercise}

\begin{solution}
    Zorn's Lemma can be applied in this case if we interpert ordering in an opposite manner.
    Therefore, we first want to show that each \emph{descending} chain of prime 
    ideals $(\idl{p_i})_i$ has a minimal element in $\Sigma$. Our claim is that 
    $\idl{p} = \bigcap_{i} \idl{p}_i$ is such a minimal element. We already know it's an ideal. To show it is 
    prime, let $xy \in \idl{p}$. Then either $xy \in \idl{p}_i$ for all $i$. In particular, this product 
    is in $\idl{p}_1$. Since $\idl{p}_1$ is prime, suppose without loss of generality that 
    $x \in \idl{p}_1$. Then $x \in \idl{p}_i$ for all $i$ and so $x \in \idl{p}$. Therefore, $\idl{p}$ is prime. 

    We conclude our ``backwards'' (but logically valid) Zorn Lemma: Since each chain has of prime ideals 
    has a minimal element, the set $\Sigma$ of all prime ideals must have minimal elements with respect to conclusion. 
    
\end{solution}

\begin{exercise}
    Let $\idl{a}$ be an ideal $\ne (1)$ in a ring $A$. Show that 
    $\idl{a} = r(\idl{a}) \iff \idl{a}$ is an intersection of prime ideals.
\end{exercise}

\begin{solution}
    Suppose $\idl{a} = r(\idl{a})$. In the ring $A/\idl{a}$, an element 
    $x  +\idl{a}$ is nilpotent if and only if $x \in r(\idl{a})$. 
    Hence, the nilpotents of $A/\idl{a}$ correspond to the elements of $r(\idl{a})$. 
    By Proposition 1.8, we know that the nilradical of $A/\idl{a}$ is the intersection 
    of all prime ideals in this ring. By the Fourth Isomorphism theorem 
    this intersection corresponds to the intersection of all prime ideals in $A$ that 
    contain $\idl{a}$. Hence $r(\idl{a})$ is the intersection of all prime ideals 
    containing $\idl{a}$, which completes this direction. 

    Conversely, suppose $\idl{a}$ is the intersection of some set of 
    prime ideals $\Sigma$ (each which obviously must contain $\idl{a}$).
    If $x \in r(\idl{a})$, then $x^n \in \idl{a}$ for some $n$. This implies that 
    $x^n \in \idl{p}$ for each $\idl{p} \in \Sigma$, so that $x \in \idl{p}$ for each $\Sigma$, and 
    hence $x \in \idl{a}$. Since we already know that $\idl{a} \subset r(\idl{a})$, this then 
    shows that $\idl{a} = r(\idl{a})$, as desired. 
\end{solution}

\begin{exercise}
    Let $A$ be a ring, $\idl{R}$ its nilradical. Show that the following are equivalent: 
    \begin{itemize}
        \item[\emph{i})] $A$ has exactly one prime ideal;
        \item[\emph{ii})] every element of $A$ is either a unit or nilpotent; 
        \item[\emph{iii}] $A/\idl{R}$ is a field.    
    \end{itemize}
\end{exercise}

\begin{solution}
    \begin{description}
        \item[\emph{i} $\iff$ \emph{ii}] Suppose $A$ has exactly on prime ideal. Then it must be a maximal ideal. 
        Additionally, since $\idl{R}$ is 
        the intersection of all the prime ideals of $A$, this means that $\idl{R}$ is maximal. 
        Therefore, every element of $A$ is either a unit or a nilpotent element. 

        Conversely, if every nonunit is nilpotent, this means that every non unit must be in $\idl{R}$. 
        Additionally, $\idl{R}$ must be maximal because there cannot be any maximal ideal containing $\idl{R}$ 
        as every nonunit is nilpotent. Hence $A$ has exactly one prime ideal. 

        \item[\emph{i} $\iff$ \emph{iii}]
        If $A$ has exactly one prime ideal, then again, $\idl{R}$ must be prime, in fact it must be maximal. 
        Therefore, $A/\idl{R}$ is a field. If conversely we are given the fact that $A/\idl{R}$ is a field, 
        this implies that $\idl{R}$ is maximal. Hence, the intersection of the prime ideals 
        of $A$ must consist of one element, so that $A$ has exactly one prime ideal.

    \end{description}
\end{solution}

\begin{exercise}
    A ring $A$ is Boolean if $x^2 = x$ for all $x \in A$. In a Boolean ring 
    $A$, show that
    \begin{itemize}
        \item[\emph{i})] $2x = 0$ for all $x \in A$
        \item[\emph{ii})] Every prime ideal $\idl{p}$ is maximal and 
        $A/\idl{p}$ is a field with two elements;
        \item[\emph{iii})] Every finitely generated ideal in $A$ is principal.
    \end{itemize} 
\end{exercise}

\begin{solution}
    \begin{itemize}
        \item[\emph{i})] Note that $(x + 1)^2 = x^2 + 2x + 1 = x + 1 \implies 2x = 0$.  
        \item[\emph{ii})] Let $\idl{p}$ be prime and consider the integral 
        domain $A/\idl{p}$. Then for all $x \in A/\idl{p}$, we have that $x^2 = x
        \implies x(x-1) = 0$. Thus we see that $x = 0$ or $1$ in $A/\idl{p}$, 
        which implies that this ring is actually a field with two elements.  
        \item[\emph{iii}] If $\idl{a}$ is finitely generated, then it must be generated by one 
        element. This is because if $a_1, a_2$ are two distinct generators, then we must be able to 
        construct the chain $(a_1) \subset (a_1, a_2)$. However, $(a_1)$ is prime and therefore 
        is maximal, so $(a_1, a_2) = (1)$. Hence, a finitely generated ideal must be principal.  
    \end{itemize}
\end{solution}

\begin{exercise}
    A local ring contains no idempotent $\ne 0,1 $.
\end{exercise}

\begin{solution}
    Let $e$ be a nonzero idempotent $\ne 0, 1$ in $A$. 
    Since $e$ is idempotent we have that 
    $e^2 - e = 0 \implies e(e-1)= 0$. Since $e \ne 0, 1$, this implies that 
    $e$ is a zero divisor. Therefore, $e$ cannot be a unit, and so $e$ and $e - 1$
    must be in the unique maximal ideal $\idl{m}$ of $A$. However, this would 
    imply that $e + (e - 1) = 1 \in \idl{m}$, a contradiction. Hence a local ring cannot 
    have a nontrivial idempotent.
\end{solution}

\begin{exercise}\emph{Construction of an algebraic closure of a field} (E. Artin).\\
    Let $K$ be a field and let $\Sigma$ be the set of all irreducible monic 
    polynomials $f$ in one indeterminate with coefficients in $K$. 
    Let $A$ be the polynomial ring over $K$ generated by indeterminates 
    $x_f$, one for each $f \in \Sigma$. Let $\idl{a}$ be the ideal of $A$ 
    generated by the polynomials $f(x_f)$ for all $f \in \Sigma$. 
    Show that $\idl{a} \ne (1)$. 

    Let $\idl{m}$ be a maximal ideal of $A$ containing $\idl{a}$, and let 
    $K_1 = A/\idl{m}$. Then $K_1$ is an extension field of $K$ in which 
    each $f \in \Sigma$ has a root. Repeat the construction with $K_1$ in place 
    of $K$, obtaining a field $K_2$ , and so on. Let $L = \bigcup_{n=1}^{\infty} K_n$.
    Then $L$ is a field in which each $f \in \Sigma$ splits completely into linear factors. 
    Let $\overline{K}$ be the set of all elements of $L$ which are algebraic over $K$. 
    Then $\overline{K}$ is an algebraic closure of $K$.
\end{exercise}

\begin{solution}
    Observe that each expression in $\idl{a}$ is one of the form 
    $c_1f_1(x_{f_1}) + \cdots + c_nf_n(x_{f_n})$
    where $c_i \in A$. As each $f_i$ is monic and irreducible, none are constant. 
    Since none are constant, and each $f_i$ is in a different variable, the above expression 
    always has degree of at least one and so it
    can never trivially evaluate to 1. Hence, $\idl{a} \ne (1)$. 

    If $\idl{m}$ is the maximal ideal of $\idl{a}$, then $K_1 = A/\idl{m}$ is a field, 
    and there is a clear injection $K \to K_1$. Hence it is a field extension of $K$, and 
    each $f \in \Sigma$ has a root in $A/\idl{m}$; namely, the root of $f \in \Sigma$ is 
    $x_f$. 
\end{solution}

\begin{exercise}
    In a ring $A$, let $\Sigma$ be the set of all ideals in which every element is a zero-divisor. 
    Show that the set $\Sigma$ has maximal elements and that every maximal element of $\Sigma$ is a prime ideal. 
    Hence the set of zero-divisors in $A$ is a union of prime ideals.
\end{exercise}

\begin{solution}
    Note that $(0) \in \Sigma$ so that $\Sigma$ is nonempty. 
    Observe also that if $\idl{a}_1 \subset \idl{a}_2 \subset \cdots$ is a chain of elements in 
    $\Sigma$, then the upper bound is given by the ideal $\idl{a} = \bigcup_{i} \idl{a}_i$, which is 
    in $\Sigma$ since it consists entirely of zero divisors. 
    Hence $\Sigma$ has maximal elements with respect to inclusion. 

    Let $\idl{p}$ be a maximal element. To show that it is prime, suppose $xy \in \idl{p}$. 
    Then this implies that $p + xy$ is a zero divisor for all $p \in \idl{p}$. Now suppose for a 
    contradiction that $\idl{p} \subset (\idl{p}, x)$ and $(\idl{p}, y)$. Then it must be the case that 
    for some $p_1, p_2 \in \idl{p}$ and $a, b \in A$, the two elements 
    $p_1 + ax $ $ p_2 + by$ are not zero divisors. 
    However, this leads to a contradiction as this would imply that 
    $(p_1 + ax)(p_2 + by) = (p_1p_2 + p_1by + p_2ax) + abxy$ is not a zero divisor. Hence we see 
    that either $(\idl{p}, x)$ or $(\idl{p}, y)$ are contained in $\idl{p}$, which implies that 
     either $x$ or $y \in \idl{p}$. So $\idl{p}$ is a prime ideal. 

    Now since the union of all the zero divisors reduces to a union of its maximal elements, 
    which are all prime, we see taht the union of all zero divisors is a union of prime ideals. 

\end{solution}

\begin{exercise}\emph{The Prime Spectrum of a Ring}\\
    Let $A$ be a ring and let $X$ be the set of all prime ideals of $A$. For each subset
    $E$ of $A$, let $V(E)$ denote the set of all prime ideals of $A$ which contain $E$. 
    Prove that 
    \begin{itemize}
        \item[\emph{i})] if $\idl{a}$ is the ideal generated by $E$, then $V(E) = V(\idl{a}) = V(r(\idl{a}))$.
        \item[\emph{ii})] $V(0) = X$, $V(1) = \varnothing$
        \item[\emph{iii})] if $(E_i)_{i \in I}$ is any family of subsets of $A$, then
        \[
            V\left( \bigcup_{i \in I}E_i \right) = \bigcap_{i \in I}V(E_i)
        \] 
        \item[\emph{iv})] $V(\idl{a} \cap \idl{b}) = V(\idl{a}\idl{b}) = V(\idl{a})\cup V(\idl{b})$ 
        for any ideals $\idl{a}, \idl{b}$ of $A$.

        These results show that the sets $V(E)$ satisfy the axioms for closed sets in a topological space. 
        The resulting topology is called the \emph{Zariski topology}. The topological space 
        $X$ is called the prime spectrum of $A$, and is written $\text{Spec}(A)$.
    \end{itemize}
\end{exercise}

\begin{exercise}
    Draw pictures of $\Spec(\mathbb{Z})$, $\Spec(\mathbb{R})$, $\Spec(\mathbb{C}[x])$, $\Spec(\mathbb{R}[x])$, $\Spec(\mathbb{Z}[x])$.
\end{exercise}

\begin{exercise}
    For each $f\in A$, let $X_f$ denote the complement of $V(f)$ in $X = \Spec(A)$. The sets $X_f$ are open. Show that they form a 
    basis of open sets for the Zariski topology, and that
    \begin{itemize}
        \item[\emph{i})] $X_f \cap X_g = X_{fg}$;
        \item[\emph{ii})] $X_f = \varnothing \iff f$ is nilpotent;
        \item[\emph{iii})] $X_f = X \iff f$ is a unit;
        \item[\emph{iv})] $X_f = X_g \iff r((f)) = r((g))$;
        \item[\emph{v})] $X$ is quasi-compact (that is, every open covering of $X$ has a finite subcovering)
        \item[\emph{vi})] More generally, each $X_f$ is quasi-compact.
        \item[\emph{vii})] An open subset of $X$ is quasi-compact if and only if it is a finite union of
        sets $X_f$.
    \end{itemize}
    [To prove (\emph{v}), remark that it is enough to consider a covering of $X$ by basic open sets $X_{f_i} (i \in I)$. 
    Show that the $f_i$ generate the unit ideal and hence that there is an equation of the form
    \[
        1 = \sum_{i \in J}g_if_i \qquad (g_i \in A)   
    \]
    where $J$ is some \emph{finite} subset of $I$. Then $X_{f_i} (i \in J)$ cover $X$. 
\end{exercise}

\begin{exercise}
    For psychological reasons it is sometimes convenient to denote a prime ideal 
    of $A$ by a letter such as $x$ or $y$ when thinking of it as a point of $X= \Spec(A)$. 
    When thinking of $x$ as a prime ideal of $A$, we denote it by $\idl{p}_x$ (logically, of course, 
    it is the same thing). Show that
    \begin{itemize}
        \item[\emph{i})] the set $\{x\}$ is closed (we say that $x$ is a ``closed point'') in 
        $\Spec(A) \iff \idl{p}_x$ is maximal; 
        \item[\emph{ii})] $\overline{\{x\}} = V(\idl{p}_x)$;
        \item[\emph{iii})] $y \in \overline{\{x\}} \iff \idl{p}_x \subset \idl{p}_y$;
        \item[\emph{iv})] $X$ is a $T_0$-space (this means that if $x$, $y$ are distinct points of $X$, then either
        there is a neighborhood of $x$ which does not contain $y$, or else there is a neighborhood of $y$ which does not contain $x$).
    \end{itemize}
\end{exercise}

\begin{exercise}
    A topological space $X$ is said to be irreducible if $X \ne \varnothing$ and if every pair 
    of non-empty open sets in $X$ intersect, or equivalently if every non-empty open set is dense in $X$. 
    Show that $\Spec(A)$ is irreducible if and only if the nilradical of $A$ is a prime ideal.
\end{exercise}

\begin{exercise}
    Let $X$ be a topological space.
    \begin{itemize}
        \item[\emph{i})] If $Y$ is an irreducible (Exercise 19) subspace of $X$, then the closure $\overline{Y}$ of $Y$
        in $X$ is irreducible.
        \item[\emph{ii})] Every irreducible subspace of $X$ is contained in a maximal irreducible
        subspace.
        \item[\emph{iii})] The maximal irreducible subspaces of $X$ are closed and cover $X$. They are
        called the \emph{irreducible components} of $X$. What are the irreducible components
        of a Hausdorff space?
        \item[\emph{iv})] If $A$ is a ring and $X = \Spec(A)$, then the irreducible components of $X$ are
        the closed sets $V(\idl{p})$ where $\idl{p}$ is a minimal prime ideal of $A$ (Exercise 8).
    \end{itemize}
\end{exercise}

\begin{exercise}
    Let $\phi: A \to B$ be a ring homomorphism. Let $X = \Spec(A)$ and $Y = \Spec(B)$. 
    If $\idl{q} \in Y$, then $\phi^{-1}(\idl{q})$ is a prime ideal of $A$, i.e., a point of 
    $X$. Hence $\phi$ induces a mapping $\phi^{*}: Y \to X$. Show that
    \begin{itemize}
        \item[\emph{i})] If $f \in A$ then ${\phi^{*}}^{-1}(X) = Y_{\phi(f)}$, and hence
        $\phi^{*}$ is continuous.
        \item[\emph{ii})] If $\idl{a}$ is an ideal of $A$, then ${\phi^{*}}^{-1}(V(\idl{a})) = V(\idl{a}^e)$.
        \item[\emph{iii})] If $\idl{b}$ is an ideal of $A$, then $\overline{{\phi^{*}}^{-1}(V(\idl{b}))} = V(\idl{b}^c)$.
        \item[\emph{iv})] If $\phi$ is surjective, then $\phi^{*}$ is a homeomorphism of $Y$ onto the closed subset 
        $V(\ker(\phi))$ of $X$. (In particular, $\Spec(A)$ and $\Spec(A/\idl{R})$ (where $\idl{R}$ is the nilradical of $A$) are naturally homeomorphic.)
        \item[\emph{v})] If $\phi$ is injective, then $\phi^{*}(Y)$ is dense in $X$. More precisely, $\phi^{*}(Y)$ is dense in $X \iff  \ker(\phi) \subset \idl{R}$.
        \item[\emph{vi})] Let $\psi: B \to C$ be another ring homomorphism. Then $(\psi \circ \phi)^{*} = \phi^{*} \circ \psi^{*}$.
        \item[\emph{vii})] Let $A$ be an integral domain with just one non-zero prime ideal $\idl{p}$ and let $K$ be the field of fractions of $A$. 
        Let $B = (A/\idl{p}) \times K$. Define $\phi: A \to B$ by $\phi(x) = (\overline{x}, x)$, 
        where $\overline{x}$ is the image of $x$ in $A/\idl{p}$. Show that $\phi^{*}$ is bijective
        but not a homeomorphism.
    \end{itemize}
\end{exercise}

\begin{exercise}
    Let $A = \prod_{i = 1}^{n} A_i$ be the direct product of rings $A_i$. 
    Show that $\Spec(A)$ is the disjoint union of open (and closed) subspaces $X_i$, where $X_i$ is 
    canonically homeomorphic with $\Spec (A_i)$.

    Conversely, let $A$ be any ring. Show that the following statements are equivalent:
    \begin{itemize}
        \item[\emph{i})] $X = \Spec(A)$ is disconnected.
        \item[\emph{ii})] $A \cong A_1 \times A_2$ where neither of the rings $A_1$, $A_2$ is the zero ring.
        \item[\emph{iii})] $A$ contains an idempotent $\ne 0, 1$. 
    \end{itemize}
    In particular, the spectrum of a local ring is always connected (Exercise 12).
\end{exercise}

\begin{exercise}
    Let $A$ be a Boolean ring (Exercise 11), and let $X = \Spec (A)$.
    \begin{itemize}
        \item[\emph{i})] For each $f \in A$, the set $X_f$ (Exercise 17) is both open and closed in $X$. 
        \item[\emph{ii})] Let $f_1 \dots, f_n \in A$ Show that $X_{f_1} \cup \cdots \cup X_{f_n} = X_f$ for some $f \in A$.
        \item[\emph{iii})]  The sets $X_f$ are the only subsets of $X$ which are both open and closed. 
        [Let $Y \subset X$  be both open and closed. Since $Y$ is open, it is a union of basic open sets $X_f$. 
        Since $Y$ is closed and $X$ is quasi-compact (Exercise 17), $Y$ is quasi-compact. Hence $Y$ is a finite union of basic open sets; 
        now use (ii) above.]
        \item[\emph{iv})] $X$ is a compact Hausdorff space.
    \end{itemize}
\end{exercise}

\begin{exercise}
    Let $L$ be a lattice, in which the $\sup$ and $\inf$ of two elements $a$, $b$ are denoted by $a \vee b$ and $a  \wedge b$ respectively. $L$ is a Boolean lattice 
    (or Boolean algebra) if
    \begin{itemize}
        \item[\emph{i})] $L$ has a least element and a greatest element (denoted by 0, 1 respectively). 
        \item[\emph{ii})] Each of $\vee$, $\wedge$ is distributive over the other.
        \item[\emph{iii})] Each $a \in L$ has a unique "complement" $a' \in L$ such that $a \vee a' = 1$ and $a \wedge a'$ = 0.
    \end{itemize}
    (For example, the set of all subsets of a set, ordered by inclusion, is a Boolean lattice.)

    Let $L$ be a Boolean lattice. Define addition and multiplication in $L$ by the rules
    \[
    a + b = (a \wedge b') \vee (a' \wedge b), \qquad ab= a \wedge b.
    \]
    Verify that in this way $L$ becomes a Boolean ring, say $A(L)$.
    Conversely, starting from a Boolean ring $A$, define an ordering on $A$ as follows: 
    $a \le b$ means that $a = ab$. Show that, with respect to this ordering, $A$ is a Boolean lattice. 
    [The $\sup$ and $\inf$ are given by $a \vee b= a+b+ab$ and $a \wedge b = ab $,and the complement by 
    $a'=1- a.$] In this way we obtain a one-to-one correspondence between (isomorphism classes of) Boolean rings and
    (isomorphism classes of) Boolean lattices.
\end{exercise}

\begin{exercise}
    From the last two exercises deduce Stone's theorem, that every Boolean lattice is
    isomorphic to the lattice of open-and-closed subsets of some compact Hausdorff topological 
    space.
\end{exercise}

\begin{exercise}
    Let $A$ be a ring. The subspace of $\Spec(A)$ consisting of the maximal ideals of $A$, with the induced topology, is called the \emph{maximal spectrum} 
    of $A$ and is denoted by $\Max(A)$. For arbitrary commutative rings it does not have the nice functorial properties of $\Spec(A)$ (see Exercise 21), \
    because the inverse image of a maximal ideal under a ring homomorphism need not be maximal.
    Let $X$ be a compact Hausdorff space and let $C(X)$ denote the ring of all real-valued continuous functions on $X$ (add and multiply functions by adding
    and multiplying their values). For each  $x \in X$, let $\idl{m}_x$ be the set of all $f \in C(X)$ such that 
    $f(x) = 0$. The ideal $\idl{m}_x$ is maximal, because it is the kernel of the (surjective) homomorphism $C(X) \to \mathbb{R}$ 
    which takes $f$ to $f(x)$. If $X$ denotes $\Max (C(X))$, we have therefore defined a mapping $\mu: X \to \tilde{X}$, namely $x \mapsto \idl{m}_x$.

    We shall show that $\mu$ is a homeomorphism of $X$ onto $\tilde{X}$.

    \begin{itemize}
        \item[\emph{i})] Let $\idl{m}$ be any maximal ideal of $C(X)$, and let $V = V(\idl{m})$ be the 
        set of common zeros of the functions in $\idl{m}$: that is,
        \[
            V = \{ x \in X : f(x) = 0 \text{ for all } f\ in \idl{m} \}.
        \]
        Suppose $V$ is empty. Then for each $x \in X$ there exists $f_x \in \idl{m}$ such that 
        $f_x(x) \ne 0$. Since $f_x$ is continuous, there is an open neighborhood $U_x$ of $x$ in $X$ 
        on which $f_x$ does not vanish. By compactness of finite number of neighborhoods, say $U_{x_1}, \dots, U_{x_n}$, cover $X$. 
        Let 
        \[
            f = f_{x_1}^2 + \cdots + f_{x_n}^2.
        \]
        Then $f$ does nto vanish at any point of $X$, hence is a unit in $C(X)$. But this contradicts $f \in \idl{m}$, hence $V$ is not empty. 
        
        Let $x$ be a point of $V$. Then $\idl{m} \subset \idl{m}_x$, hence $\idl{m} = \idl{m}_x$ because $\idl{m}$ is maximal. 
        Hence $\mu$ is sujrective.
        
        \item[\emph{ii})] By Urysohn's lemma (this is the only non-trivial fact required in the argument) the continuous functions separate the 
        points of $X$. Hence $x \ne y \implies \idl{m}_x \ne \idl{m}_y$ and therefore $\mu$ is injective.
                
        \item[\emph{iii})] Let $f \in C(X)$; let 
        \[
            U_f = \{ x \in X : f(x) \ne 0\} 
        \]
        and let 
        \[
            \tilde{U}_f = \{\idl{m} \in \tilde{X} : f \not\in \idl{m}\}.
        \]
        Show that $\mu (U_f) = \tilde{U}_f$. The open sets $U_f$ (resp. $\tilde{U}_f$ ) form a basis of the topology of $X$ (resp. $\tilde{X}$) 
        and therefore $\mu$ is a homeomorphism.

        Thus $X$ can be reconstructed from the ring of functions $C(X)$.
    \end{itemize}
\end{exercise}

\begin{exercise}\emph{Affine Algebraic Varieties}.
    Let $k$ be an algebraically closed field and let 
    \[
        f_{\alpha}(t_1, \dots, t_n) = 0
    \]
    be a set of polynomial equations in $n$ variables with coefficients in $k$. 
    The set $X$ of all points $x = (x_1, \dots, x_n) \in k^n$ which satisfy these equations is an 
    \emph{affine algebraic variety}.

    Consider the set of all polynomials $g \in k[t_1, \dots, t_n]$ with the property that $g(x) = 0$ for all 
    $x \in X$. This set is an ideal $I(X)$ in the polynomial ring, and is called the ideal of the variety $X$. 
    The quotient ring
    \[
        P(X) = k[t_1, \dots, t_n]/I(X)
    \]  
    is the ring of polynomial functions on $X$, because two polynomials $g, h$ define the same polynomial function 
    on $X$ if and only if $g - h$ vanishes at every point of $X$, that is, if and only if $g - h \in I(X)$.
    
    Let $\eta_1$ be the image of $t_i$ in $P(X)$. The $\eta_i (1 \le i \le n)$ are the \emph{coordinate functions} on $X$: 
    if $x \in X$, then $\eta_i(x)$ is the $i$th coordinate of $x$. $P(X)$ is generated as a $k$-algebra by the coordinate functions, and is called the 
    \emph{coordinate ring} (or affine algebra) of $X$.
    
    As in Exercise 26, for each $x \in X$ let $\idl{m}_x$, be the ideal of all $f \in P(X)$ such that $f(x) = 0$; 
    it is a maximal ideal of $P(X)$. Hence, if $X= \Max(P(X))$, we have defined a mapping $\mu: X \to \tilde{X}$,  namely $x \mapsto \idl{m}_x$.
    
    It is easy to show that $\mu$ is injective: if $x \ne y$, we must have $x_i \ne y_i$ for for some $i(1 \le i \le n)$, 
    and hence $\eta_i - x_i$ is in $\idl{m}_x$ but not in $\idl{m}_y$, so that $\idl{m}_x \ne \idl{m}_y$. 
    What is less obvious (but still true) is that $\mu$ is \emph{surjective}. This is one form of Hilbert's Nullstellensatz (see Chapter 7).
\end{exercise}

\begin{exercise}
    Let $f_1, \dots, f_m$  be elements of $k[t_1, \dots, t_n]$. They determine a \emph{polynomial mapping} 
    $\phi: k^n \to k^m$ if $x \in k^n$, the coordinates of $\phi(x)$ are $f_1(x), \dots, f_m(x)$.
    
    Let $X$, $Y$ be affine algebraic varieties in $k^n$, $k^m$ respectively. A mapping $\phi: X \to Y$ is said to be regular if $\phi$ is 
    the restriction to $X$ of a polynomial mapping from $k^n$ to $k^m$.
    If $\eta$ is a polynomial function on $Y$, then $\eta \circ \phi$ is a polynomial function on $X$. Hence $\phi$ 
    induces a $k$-algebra homomorphism $P(Y) \to P(X)$, namely $\eta \mapsto \eta \circ \phi$. Show that in this way we obtain a 
    one-to-one correspondence between the regular mappings $X \to Y$ and the $k$-algebra homomorphisms $P(Y) \to P(X)$.
\end{exercise}


\end{document}