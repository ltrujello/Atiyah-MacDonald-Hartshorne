\documentclass[10pt]{amsart}
\usepackage[margin=0.7in]{geometry}
\usepackage{graphicx}
\usepackage{amsmath, amssymb, amsthm} 
\usepackage{lmodern}
\usepackage[T1]{fontenc}
\usepackage{fancyhdr}
\usepackage[usenames,dvipsnames]{xcolor} % If using tikz, put \usepackage{tikz} after this
\usepackage{pdfcolmk}

% Our commands + environments are in style.cls. 
\usepackage{../atiyah-macdonald-style}

\begin{document}
\header{Chapter 2}

\begin{exercise}[Exercise 17]
\end{exercise}

\begin{solution}
    Recall from Exercise 14 that 
    \[\lim_{\to} M_i \cong C/D = \bigoplus_{i \in I}M_i/D,\]
    where $D$ is the submodule of $C = \bigoplus_{i \in I}M_i$ generated by all elements of the form $x_i - \mu_{ij}(x_i)$. Also recall that $\sum M_i$ is the set of all finite sums.
    Consider the map 
    \[\phi: \bigoplus_{i \in I}M_i \to \sum M_i.\]
    For the first isomorphism $\lim_{\to} M_i \cong \sum M_i$, it suffices to show that $\ker \phi  = D$. 
    \begin{enumerate}
        \item Clearly $\phi(x_i - \mu_{ij}(x_i)) = 0$. So $D \subseteq \ker \phi$.
        \item Conversely, let $x = (x_i)_{i \in I} \in C$. Then $x_i \neq 0$ for finitely many $i \in I$. Therefore, there exists $k \in I$ such that if $x_i \neq 0$, then $M_i \subseteq M_k$. Since $\phi(x) = 0$, we must have
        \[
        x_k = \sum_{i \in I, i \neq k, x_i \neq 0}\mu_{ik}(-x_i).
        \]
        Therefore, we may write 
        \[
        x = (x_i)_{i \in I} = x_k + \sum_{i \in I, i \neq k, x_i \neq 0}x_i = \sum_{i \in I, i \neq k, x_i \neq 0} (\mu_{ik}(-x_i) + x_i) \in D.
        \]
    \end{enumerate}

    The second isomorphism $\sum M_i \cong \bigcup M_i$ follows from the fact that a finite sum of elements in various $M_i$'s may be rewritten as a sum of elements in some $M_k$.

\end{solution}


\end{document}