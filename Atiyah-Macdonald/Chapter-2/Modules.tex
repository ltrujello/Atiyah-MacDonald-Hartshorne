\documentclass[10pt]{amsart}
\usepackage[margin=0.7in]{geometry}
\usepackage{graphicx}
\usepackage{amsmath, amssymb, amsthm} 
\usepackage{lmodern}
\usepackage[T1]{fontenc}
\usepackage{fancyhdr}
\usepackage[usenames,dvipsnames]{xcolor} % If using tikz, put \usepackage{tikz} after this
\usepackage{pdfcolmk}

% Our commands + environments are in style.cls. 
\usepackage{../atiyah-macdonald-style}

\begin{document}
\header{Chapter 2}

\begin{exercise}[Exercise 17]
\end{exercise}

\begin{solution}
    Recall from Exercise 14 that 
    \[\lim_{\longrightarrow} M_i \cong C/D = \left(\bigoplus_{i \in I}M_i \right)/D,\]
    where $D$ is the submodule of $C = \bigoplus_{i \in I}M_i$ generated by all elements of the form $x_i - \mu_{ij}(x_i)$. Also recall that $\sum M_i$ is the set of all finite sums.
    Consider the map 
    \[\phi: \bigoplus_{i \in I}M_i \to \sum M_i.\]
    For the first isomorphism \[\lim_{\longrightarrow} M_i \cong \sum M_i,\] it suffices to show that $\ker \phi  = D$. 
    \begin{enumerate}
        \item Clearly $\phi(x_i - \mu_{ij}(x_i)) = 0$. So $D \subseteq \ker \phi$.
        \item Conversely, let $x = (x_i)_{i \in I} \in C$. Then $x_i \neq 0$ for finitely many $i \in I$. Therefore, there exists $k \in I$ such that if $x_i \neq 0$, then $M_i \subseteq M_k$. Since $\phi(x) = 0$, we must have
        \[
        x_k = \sum_{i \in I, i \neq k, x_i \neq 0}\mu_{ik}(-x_i).
        \]
        Therefore, we may write 
        \[
        x = (x_i)_{i \in I} = x_k + \sum_{i \in I, i \neq k, x_i \neq 0}x_i = \sum_{i \in I, i \neq k, x_i \neq 0} (\mu_{ik}(-x_i) + x_i) \in D.
        \]
    \end{enumerate}

    The second isomorphism $\sum M_i \cong \bigcup M_i$ follows from the fact that a finite sum of elements in various $M_i$'s may be rewritten as a sum of elements in some $M_k$.

\end{solution}

\newpage


\begin{exercise}[Exercise 21]

\end{exercise}

\begin{solution}
    By Exercise 14, $A$ is a $\Z$-module and the mappings $\mu_i: A_i \to A$ are $\Z$-module homomorphisms. It remains to endow $A$ with a multiplicative structure, check that it is compatible with the addition in $A$, and check that the mappings $\mu_i: A_i \to A$ respect multiplication.

    Let $x \in A$. Recall from Exercise 15 that there exists $i \in I$ and $x_i \in A_i$ such that $x = \mu_i(x_i)$. Therefore, given $x, y \in A$, there exists $i$ and $x_i, y_i \in A_i$ such that $x = \mu_i(x_i)$ and $y = \mu_i(y_i)$. Thus, define $xy = \mu_i(x_iy_i)$. 

    \begin{claim}
    This multiplication is well-defined. 
    \end{claim}
    \begin{proof}
    Since the system is directed, it suffices to show that if $i \leq k$ and $x = \mu_i(x_i) = \mu_k(x_k)$, $y = \mu_i(y_i) = \mu_k(y_k)$, then $\mu_i(x_iy_i) = \mu_k(x_ky_k)$. 

    By Exercise 15, $m \in \ker \mu_k$ if and only if there exists $j \geq k$ such that $\mu_{kj}(m) = 0$. In other words, \[\ker \mu_k = \bigcup_{k \leq j} \ker \mu_{kj}.\] Since $\mu_{kj}$ is a ring homomorphism, the right hand side is a union of ideals in $A_k$, and so the left hand side is also an ideal of $A_k$.

    Since $\mu_k(\mu_{ik}(x_i) - x_k) = \mu_i(x_i) - \mu_k(x_k) = x-x=0$, we have that $\mu_{ik}(x_i) - x_k \in \ker \mu_k$. Similarly, $\mu_{ik}(y_i) - y_k \in \ker \mu_k$.
    Now we are ready to show the desired equality.
    \begin{align*}
        \mu_k(x_ky_k) - \mu_i(x_iy_i) &= \mu_k(x_ky_k-\mu_{ik}(x_i)\mu_{ik}(y_i)) \\ 
        &=
        \mu_k((x_k-\mu_{ik}(x_i))y_k + 
        \mu_{ik}(x_i)(y_k-\mu_{ik}(y_i))) \\ 
        &= \mu_k((x_k-\mu_{ik}(x_i))y_k) + \mu_k(\mu_{ik}(x_i)(y_k-\mu_{ik}(y_i))) \\ 
        &= 0,
    \end{align*}
    where at the last step, we used the property that $\ker \mu_k$ is an ideal.
    \end{proof}

    It's not hard to check that this multiplication respects addition, and it follows directly from our definition of the multiplication structure of $A$ that the mappings $\mu_i: A_i \to A$ are ring homomorphisms. 

    \begin{claim} 
        If $A = 0$, there exists $i \in I$ such that $A_i = 0$.
    \end{claim}
    \begin{proof}
        Choose some $j \in I$ and suppose that $A_j \neq 0$. Let $1_j$ be the identity element of $A_j$. If $A = 0$, we have $\mu_{j}(1_j) = 0$. By Exercise 15, there exists $k \geq j$ such that $\mu_{jk}(1_j) = 0$. Since $\mu_{jk}$ is a ring homomorphism, this is only possible if $A_k = 0$.
    \end{proof}
\end{solution}

\newpage

\begin{exercise}[Exercise 22]
\end{exercise}

\begin{solution}

\end{solution}

\end{document}